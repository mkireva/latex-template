% Example Section - A Beautiful Addition to Your Book
\raggedbottom

\chapter{Gespräche mit dem Geist Gottes}

% Content with elegant formatting following the book's style

\section{Erstes Gespräch -- Anleitung}


Ich weiß die Wahrheit an sich, und wenn ich sie ausspreche oder ihr eine sichtbare Form verleihe, wird sie dir, um dessen Willen alles gesprochen wird, etwas nutzen. Diese Wahrheit ist groß. Aber wie verhält es sich mit deinem Glauben in Bezug auf Gott? Bist du bereit, mich zu hören und das zu befolgen, was ich dir jetzt sagen werde? Wenn es so ist, dann wird Gott in Seinen Absichten nicht zögern, dir das zu offenbaren, was du wissen sollst und was für deinen Geist nötigt ist. Die Kraft liegt in der Langmut, aber sie hat auch ihre Grenzen. Denn, wenn es jemanden gibt, der sich langmütig nennen kann, dann ist es Gott, dessen Barmherzigkeit auf ewig sein wird. Wenn jedoch die göttliche Langmut nicht so groß wäre, wie würdest du darauf antworten? Ich weiß, dass man alle Dinge anhalten oder aber bis zu einem gewissen Grad laufen lassen kann, aber außerhalb dieser Grenze, die Gott setzt, ist alles ein Risiko, und das Böse kann wen auch immer ereilen. Ja, das Böse, sage ich, das unauslöschliche Feuer der Hölle, das nach Opfern und Zerstörung trachtet. Du siehst und begreifst jedoch, dass die Macht des Ewigen bei der Ausführung jeder Sache und Tat nötig ist. Denn was für einen Auftrag wird jemand haben, der eine Tat verrichten wollte, um die anderen von der Aufrichtigkeit zu überzeugen, wenn diese Tat nicht den Abdruck der Hand Gottes hätte? Im Auftrag Gottes selbst besteht die Größe dieser Arbeit, die mit Seiner Unterstützung begonnen wurde. Zu Beginn hätte sie möglicherweise nicht diesen Augenschein erweckt, aber im Laufe der Zeit, in der die Wahrheiten auf ihren richtigen Weg gesetzt wurden, wird schließlich gesagt werden, dass Gott hier gewirkt hat. Derartige Taten Gottes können nicht mit der vergänglichen Zeit gemessen werden, sondern mit der Ewigkeit, in der alles, was sich in Vollkommenheit entwickelt, seinen göttlichen Ursprung beweisen wird.

Nun -- in der Tat. Was für einen Nutzen wirst du haben, wenn du an deiner Berufung zweifelst oder bei deiner Arbeit, die dir gegeben wurde, um sie zu verrichten, zauderst? Deine Zweifel oder dein Zaudern, die du empfindest, können nicht die Frucht deiner Seele sein. Sie können deinem Verstand von manchen Umständen von außen aufgedrängt worden sein, für welche Gott nicht verantwortlich ist. Aber damit du die Kraft hast, sie aufzuheben, sollst du deine Seele in eine unmittelbare Beziehung zu diesem lebendigen Gott, dem Herrn des Lichtes, setzen und Ihm auf eine unmittelbare Weise deine Bedürfnisse und Absichten mitteilen und sehen, wie Seine Billigung sein wird. Denn wenn ein Gedanke oder ein Keim, der sich in deiner Seele eingenistet hat, göttlichen Ursprungs ist, wird er, in Berührung mit seinem Ursprung oder Grund gekommen, im Nu seinen Ursprung und seine Natur zeigen. Denn jeder Keim wird an seinem Ort erfahren oder befruchtet. Der Keim eines jeden Lebens erfordert seine passende Gebärmutter. Wenn die Bedingungen folglich so sind, dass sie die edlen Mittel für das Erreichen einer edlen Tat erfordern, so ist es an der Zeit, dass sie gesucht werden, wo immer sie auch sein mögen. Unter diesen Bedingungen wird der Gehorsam gegenüber dem Willen Gottes vollzogen. Denn Gott will, dass wir nicht nur Sein Werk verrichten, sondern es auch gut verrichten. Das ist der Grund, warum Gott seine Auserwählten einer langen und schwierigen Prüfung unterzieht, damit sie sich an seinem Werk, mit dem Er sie beauftragen wird, gewöhnen, auf dass sie es so vollziehen, wie Er es selbst vollziehen würde. Und da das Werk Gottes in dieser Welt von einem in der Vollkommenheit des wahren Lichtes unerfahrenen und unerzogenen Menschen vollbracht wird, wird das zur Ursache, seinen Gang zu verlangsamen, um dem Heiligen Geist Zeit zu geben, die unverbesserlichen Dinge zu berichtigen. Darin nämlich werden sich Gottes Worte erfüllen -- in seinem Wort, dass alles, was diejenigen ereilt, die ihn lieben, ihnen zu ihrem Wohl verhilft. Auf diese Weise steht Er ihnen nämlich bei, um sie wissen zu lassen, dass sie in Zukunft nicht so handeln sollen wie in der Vergangenheit. Und von hier aus sagen wir, dass das Leben nichts anderes sei als Erfahrung. Ja, Erfahrung, von der gesagt werden kann, dass sie aus einer Reihe von Fehlern gewonnen wurde, unserem Wissen oder unserer Unwissenheit entsprechend. Aber immer soll alles innerhalb seiner Grenzen geschehen. Die Fehler verwandeln sich nur dann in Nutzen, wenn die Liebe hineinkommt, um als Kraft zu arbeiten. Hier ist dasselbe Gesetz wirksam, das in der Natur arbeitet, alle schädlichen Elemente umkehrt, damit sie für den Nutzen der Menschheit tätig sind. 

Wisse, dass dein Leben von vielen kleinen Dingen abhängt, mit denen Gott dich in Verbindung gesetzt hat, und die nicht vernachlässigt werden können, ohne dass du selbst einen Schaden davon trägst. Wenn du derartigen Gefahren ausgesetzt bist, ist es nötig, dass du ständig unter der Leitung des immer wachsamen Auges Gottes stehst. Sieh, heute beginne ich, dir manche Dinge nacheinander zu offenbaren, die dir nützlich sein werden; sei folglich bereit, mir bis zum Schluss zuzuhören. Denn ich werde zu dir wie zu einem vernünftigen Menschen sprechen. 

Nun ist das Leben bezüglich des Aufstiegs langwierig und gleichzeitig schwierig. Jeder Schritt nach oben erfordert Mühen, Arbeit und Beständigkeit. Die höheren Güter lassen sich nicht leicht erlangen. Man sagt, wer siegt, dem werde die Krone des Lebens verliehen. Und das ist richtig und wahr. Nicht alle können das Reich Gottes erben, denn bei vielen besteht nicht den Wunsch, für das Gute eine der Buße würdige Frucht zu bringen.\footnote{Vgl. Mt 3,8.} Weil Gott diese innere Veranlagung sieht, bestimmt Er nur diejenigen, in denen die Bereitschaft und der Wunsch besteht, das Wertvollste zu opfern, damit nur sie die wertvolle Perle des Reiches Gottes erlangen. Aber alles, was ich dir bis jetzt gesagt habe, soll dich nicht beunruhigen, denn die unmöglichen Dinge für den Menschen sind möglich für Gott. Mein Wort ist für dich, der du dich von nichts stören lassen darfst, denn alles steht für dich von Gott zum Guten bereit. Und wer kann, so wie Gott dich liebt, sich dem widersetzen? Und wer kann, wenn Er dir seine Gnade zeigt, dem widerstreben? Denn Gott ist kein Mensch -- denn Er ändert sich nicht. Noch ist Er Sohn des Menschen -- denn Er wendet sich von seinen Absichten nicht ab. Er hat Gewalt und Macht, alles zu tun, was Er will und denkt, und niemand kann sich Ihm widersetzen. Aber es ist notwendig, alle göttlichen Wege zu kennen und sie mit der Ganzheit deines Herzens zu wahren. Darin besteht die Vollkommenheit -- in der Erkenntnis des Willens des einen und wahren Gottes, des Herrn und Erlösers. Groß sind seine Wohltaten, die er für diejenigen bereitstellt, die Ihn lieben. Dort oben im Himmel gibt es heilige Wohnstätten, die für die Gerechten, für die Kinder Gottes vorbereitet wurden.

Ich habe dir noch mitzuteilen, dass Gott in deiner Seele schon eine große Veränderung vorbereitet, die nur den Söhnen Gottes eigen ist. Deine Seele wird bald wie durch neue Augen auf die Welt und auf alles schauen, was um dich herum getan wird. Es wird für dich nicht mehr verwunderlich sein, diese verborgenen Geheimnisse zu verstehen, die jetzt deinen Verstand verwirren. Kann ich offener zu dir sprechen und dir das zu verstehen geben, womit ich beauftragt worden bin, es dir zu offenbaren? Denn die Dinge haben ihre Bedeutung nur für denjenigen, der sie versteht und den Sinn in seinem Herzen trägt. Denke nicht, dass der Himmel deinen Mühen gegenüber gefühllos ist. Jeder Schritt, den du aufwärts und vorwärts vollziehst, erfreut alle dort oben. 

Wir schlafen nicht, sondern wir wachen immer; wir sind nicht untätig, sondern wir arbeiten immer. Unsere Freude liegt im Befolgen unserer Pflicht. Und was für eine Freude fühlen wir, wenn wir aufgerufen sind, auch die kleinste und unbedeutendste Tat auszuführen, was euch viele Male empört. Es ist für uns weder schwer noch fühlen wir Trauer, wenn wir vorbeikommen, um sowohl den höchsten als auch den kleinsten Diensten nachzugehen. Das Einzige, was unseren Geist manchmal betrübt, ist, wenn wir sehen, dass ihr den entgegengesetzten Weg der Liebe eingeschlagen habt. Wenn die göttliche Langmut nicht wäre, hätten wir uns längst mit den Sündern und ihren Sünden auseinandergesetzt. Aber die Liebe erfordert Geduld, bis das Glas der Geduld überläuft, und dann kommt es zur Gesetzlosigkeit. Merkt euch, dass der Himmel weder die kleinste Sünde dulden noch sie verdecken kann. Jede Sünde soll bestraft und berichtigt werden. Der, der sie begangen hat, soll sie einsehen, büßen und erkennen, dass die Sünde etwas der geistigen Natur und den Söhnen Gottes Entgegengesetztes ist. Die Sünde ist keine Frucht des Unwissens, wie einige meinen, sondern die Frucht der Hölle, die Frucht der ungehorsamen Mächte der Hölle. Eine Sünde kann jeder begehen, wenn er Gott leugnet und beginnt, Seinen Willen in sich und in seinem Herzen zu missachten. Deshalb zieht die Verzweiflung im Leben immer die Sünde nach sich. Denn bevor der Mensch sündigt, muss er Gott geleugnet haben und denken, Gott sei nicht allgegenwärtig und allwissend, und er könne ein Verbrechen begehen, ohne bemerkt zu werden. Siehe, die Wurzel der Sünde. Wenn ein Mensch seine Augen und Herz schließt und in sich sagt: "`Mich wird niemand sehen, ich muss mich selbst um alles und um mich selbst kümmern."' --, dann hat er die Sünde in seinem Herzen begangen. Er erschafft einen Weg, einen Plan für sein Leben, den er mit allen unverzeihlichen Mitteln zu verwirklichen beginnt. Solch ein Werk, wie erfolgreich es auch sein mag, hat nicht den Segen Gottes. Früher oder später wird das Ende tödlich sein. In der Welt hat nur Gott ein Recht darauf, das Leben so zu erschaffen und zu regeln, wie Er es will. Jeder, der sagt, dass er Gott nicht kennt und seinen guten Willen nicht erfüllen will, ist vom Teufel geboren worden. Einen solchen Menschen, wer immer er auch sei und in welchem Zustand auch immer er sich befinden vermag, wird sich an dem Ort und dem Raum befinden, den nur die Geister des Unterhimmels bewohnen. Die Sünde im Leben ist zweifach; wenn eine Seele alles nach dem Wunsch ihres Herzens tut und Gott leugnet, nämlich dass Er der Herrscher ist, und wenn eine Seele den Geist in sich hindert, das Gute, das er tun könnte, zu tun, und dadurch Gott daran hindert, es durch den Geist zu tun. 

Ich sehe jetzt, dass meine Äußerungen dich beunruhigen, da du befürchtest, möglicherweise in ein Verbrechen verwickelt zu sein. Aber hab' keine Angst. Aus dir selbst heraus sollst du dich erleuchten und die Wahrheit annehmen wie das Licht in deiner Seele, das dir zeigt, was gut und böse ist. Die Sünde aber beginnt im Herzen, dann geht sie zu den Gedanken und schließlich zum Willen über. Und nun wird das, was verborgen war, offenbar, d. h., es beginnt sich zu vollziehen. Deshalb besteht der Geist Gottes darauf, zuerst das Herz, von dem alle schlechten Gedanken kommen, in der Hand zu haben.

25. Juni 1900


Die Worte der Wahrheit werden in diesen und in den kommenden Tagen in Erfüllung gehen, und alle Worte Gottes, die er zu all seinen Dienern, Propheten und Priestern gesprochen hat, werden bestätigt werden. Die kommenden Zeichen des Jahrhunderts sind die vorletzten vor der Zeit des Jüngsten Gerichtes, das die Erde und alle Lebenden und Toten auf ihr besuchen wird. Das Verderben und der Unglaube, die die Sünde in sich tragen, werden augenblicklich vernichtet werden. Der Herr wird die Unreinen zugrunde richten und sein Volk zu sich rufen. Es ist an der Zeit, dass alle auf ewig in der Wahrheit Seiner Gegenwart getröstet werden. Gesegnet seien von nun an alle, die Gott den Herrn lieben. Frieden soll über ihnen sein. Friede allen, die auf Ihn vertrauen. So spricht der Herr, der Erlöser. 



30. Juni 1900

\section{Zweites Gespräch -- Das Herz und Gott}

Ich habe dir im vorigen Gespräch gesagt, dass das Herz unter der Leitung des Geistes Gottes stehen soll, da von ihm die Fügungen des Lebens abhängen. Das ist eine Wahrheit. Das Herz, welches das Zentrum des geistigen Lebens ist, kann, wenn es nicht richtig gesteuert wird, die Seele zerstören, indem es all ihre Kräfte verausgabt und eine innere Zerstörung hervorruft, die Verzweiflung, Erbitterung und Hass jedem Leben gegenüber genannt wird. Das Herz, das alles ausgegeben, nichts gespart und nichts als Ersatz erlangt hat, wird sich nach den Gesetzen des Lebens selbst -- alles inneren Guten beraubt -- notwendigerweise in Armut und Dürftigkeit befinden, und da es nicht gelernt hat, solche Entbehrungen zu ertragen, entscheidet es, sich selbst zu zerstören, statt die Nachteile zu ertragen. Nun räumt der Mensch in einem solchen Fall den bösen Geistern Platz in sich ein und erlaubt ihm, ihn zu erobern und ihn weit von der wahren Freiheit hinweg zu führen. Das ist das Unglück der heutigen Welt, dass sie in geistiger Hinsicht mehr ausgibt als sie einnimmt. So entsteht eine Krise in ihrem Leben. Die Sitten verfallen, die guten Gewohnheiten verlieren ihren Zweck, die leuchtenden Gedanken verschwinden, die guten Gefühle werden verdorben und die gute Saat, die Frucht erbringen soll, verliert sich unter den Dornen und Disteln, die um den Menschen herum gewachsen sind. Das ist die Ursache, die den Menschen selbst verdirbt -- sein Nichtwollen, das Gute für seine Seele zu erfahren. Ich denke, dass es mit dir nicht so sein soll. In allem sollst du um das Gute kämpfen, es täglich und stündlich tun, jede Stunde -- niemals sollst du geizen, unabhängig davon, unter welchen Bedingungen du dich befindest, das Gute zu tun. 

Du sollst immer auf dem Weg des Lichtes gehen. Keine Sünde darf dich verführen und kein schlechtes Gefühl irreführen, denn solche Dinge sind zerstörerisch für deine Vervollkommnung. Weißt du, dass auch die kleinste Sünde und das kleinste Verbrechen die Zerstörung des Lebens nach sich ziehen kann? Zweifle nicht daran. Du hast es erfahren, und es ist nicht nötig, dass ich dich davon überzeuge. Wir sollen die Wahrheit sprechen und sie nicht verbergen. Und deshalb bin ich gekommen, mit dir über die Wahrheit zu sprechen, damit du sie verstehst und begreifst und sie als Leiterin in deinem Leben hast. Die Zweifel, die ständig deine Seele und deinen Verstand bedrängen, verhindern vorübergehend deinen Erfolg. Ich habe dir so viele Male gesagt, dass du nicht alleine in dieser Welt bist und dass dein Leben von Gott abhängt und Er es ständig regelt. Deine Erlösung wurde vor langer Zeit vorbereitet, lange bevor du es wusstest. Von dir wird verlangt, nur das zu empfangen, was für dich als Gabe vorgesehen ist. Vor allem sollst du wissen, dass dich Gott auf die eine oder andere Weise in dieser Welt immer begleitet. Es ist nicht wichtig zu wissen, auf welche Weise, sondern zu wissen, dass Er auf deinem Weg Freund mit dir ist. Daran zu glauben, ist ein großer Segen für die Seele. Denn sie soll mit Gott verbunden sein und Ihm immer gehören. Keine zweitrangigen Gedanken dürfen deinen Verstand davon ablenken. Was du auch immer in dieser Welt zu besitzen vermagst, ist vergänglich, und die wesentlichen und guten Dinge sind ewig. Sie sind die Dinge des künftigen Lebens. Sie sind der ewige Reichtum, der deiner Seele gehört. 

Indem ich dir diese Dinge diktiere, sage ich dir, dass der Zweifel und die innere Unentschiedenheit und Unentschlossenheit eine Schwäche und eine Last für ein Leben wie das deinige sind. Verstehe mich, die Liebe verlangt nach Opfern und Selbstopfern. Wenn du dich selbst nicht vollkommen verleugnest, kannst du kein Schüler deines Herrn sein. Ich sehe hier, dass es viele Dinge in deinem Herzen gibt, die du zurückweisen sollst. Weißt du, wie viele leere Gedanken und Wünsche du hast? Sage dich davon los, das Leben besteht nicht in ihnen. Ich möchte zu dir klar und ohne Umschweife sprechen. Weißt du, warum du wegen der Drangsale so versehrt wurdest und warum du so viel leidest? Um zu lernen, dass alles von Gott kommt. Du weißt sehr gut, dass du von vielen Dingen geträumt hast, die nicht in Erfüllung gegangen sind; du hast dich dazu entschlossen, viele Dinge zu tun und konntest sie nicht tun. Und warum? Siehe, das ist es -- weil sie leer und eitel waren und weil es Gott nicht gefallen hat, dass du dafür die Zeit deines Lebens vergeudest. Er hat sie verboten, weil Er dich geliebt und dein Wohl gewünscht hat wie kein anderer. Du hast all das als einen Schlag des Missgeschicks betrachtet, als ein Unglück deines Herzens. Aber denk nach -- wenn es dir erlaubt worden wäre, alles zu verrichten, wo wärst du jetzt? Nein, nein, danke! Dein Leben hat eine höhere Bestimmung. Dein Schicksal ist von Gott für etwas Besseres bestimmt worden. Du hast nichts mit den Feinden Gottes gemein und mit ihrem Irrsinn, auch wenn sie sich bemüht haben, dich in ihren Netzen zu verwickeln und Dir den Preis zu rauben, der dir zufallen wird. Aber Gott hat alle ihre Absichten und Vorhaben vereitelt. Er hat dich immer in Obhut genommen und war in den gefährlichsten Minuten deines Lebens deiner Seele nah.

Ich spreche zu dir -- dein Freund, der zu dir mit der Absicht gekommen ist, dir diese Dinge zu sagen, die du von Gott verlangt und erbeten hast. Und ich freue mich heute, dass ich bei dir bin, dass ich zu dir von Angesicht zu Angesicht sprechen kann. Weißt du warum? Weil deine Seele jetzt frei ist, und du mit Geduld die Gaben und den Segen Gottes erwartest. Ich habe gesehen, wie oft der Teufel Unkräuter in dein Leben gesät hat, aber du hast sie zeitig erkannt und sie mit dem Geist zertreten, damit sie nicht hervorsprießen konnten. Aber der Herr, der all deine guten Mühen gesehen hat, hat sie gesegnet, Er hat das Böse sofort an seinem Platz vernichtet, hat es zurückgelassen wie ein Raubtier ohne Zähne und Nägel, damit es dir zum Guten dient. Und jetzt schreite ich zu diesem Ort, um dir zu helfen, um dich ständig zu festigen. Die Tage, die für dich kommen, sind Tage des Wohls und des Segens.

Gesegnet sei der Herr um seiner großen Barmherzigkeit und Güte willen! Stehe auf und gehe mit diesem Gedanken, dass Gott nah ist, zu Bett. Ich sage dir, langweile dein Herz nicht. Jetzt ist die Zeit, um munter zu sein. Der Herr wird sein Werk ausführen. Alle seine Mittel sind vorbereitet. Ich sage zu dir: Für ihn gibt es nichts Unmögliches. Er wird alles erfüllen. Siehe, mit meinem Geist, der für die Welt unbegreiflich ist, werde ich alles für dich überwinden. Meine Macht ist in meinem Wort, und mein Segen in der Barmherzigkeit. Und lege all meine Worte in dein Herz und beginne keine Verhandlungen über die Zukunft. Lege sie in dein Herz und betrachte sie, bis sie zu leben beginnen. Denn nicht nur durch das Brot ist der Mensch lebendig, sondern durch jedes Wort, das aus dem Munde Gottes kommt.

Gütig bist du, mein Herr Gott und Dir gebührt es, dass alle Dich lobpreisen und zu Dir beten und sich im Geist und in der Wahrheit, die Du in ihre Herzen und Geister hineingelegt hast, vor Dir verneigen. Verherrlicht werde Dein Name, Dir sei Ruhm, der Du uns von unseren Feinden erlöst und uns an einen ungefährlichen Ort gestellt hast! Wir schauen Dein Antlitz und werden uns immer freuen und Dir ein neues Lied singen. Wir werden mit großem Jauchzen frohlocken, dass du Zion erlöst und die neue Stadt Jerusalem als unsere Wohnstätte befestigt hast; in ihr werden wir Dir immer dienen.



30. Juni 1900




\section{Drittes Gespräch -- Nahrung und Wort}

Im vorigen Gespräch habe ich zu dir gesagt, dass der Mensch nicht nur vom Brot, sondern von jedem Wort, das aus dem Mund Gottes kommt, lebt. Das ist eine Wahrheit. Jede Seele hat das Bedürfnis, sich mit der vernünftigen Milch der Wahrheit zu ernähren, die das belebende Feuer ist. Diese Nahrung ist für das innere Leben so erforderlich wie die Nahrung für dieses vergängliche Leben. Mit ihrer belebenden Wirkung erlangt die Seele Fülle, die zur Vollkommenheit führt, wie die Nahrung für den Körper nicht nur das Ziel hat, den Körper zu unterstützen und ihm Kraft zu liefern, sondern ihm auch zu helfen, seine Gebrechen zu heilen, die zufällig oder unvorhergesehen in seinen Organismus gelangt sind. Die gesunde Nahrung hat die Bestimmung, wenn der Organismus des Körpers selbst nicht verdorben ist, die ungesunden Substanzen zu vertreiben, indem sie sie durch gesunde ersetzt. So ist es auch mit dem Wort Gottes. Es bezweckt, die Seele von allem Lügenhaften und Trügerischen zu befreien und sie mit der heilsamen Lehre des Lebens zu erfüllen, die Licht und Vergnügen für das lebendige Herz ist. Wie der Mensch bei der vergänglichen Nahrung ein Bedürfnis nach ihr empfinden soll, bevor er sie zu sich nimmt, so soll er es auch bei der geistigen. Er soll jedoch nicht nur das Bedürfnis empfinden, sondern seine Nahrung auch auswählen, damit er nicht unbeabsichtigt eine bestimmte Nahrung zu sich nimmt, die ihm schadet. So wie beim Essen der wesentlichen Nahrung die Vorsehung mittels ihrer feststehenden Gesetze in der Welt Vorsichtsmaßnahmen bei den Menschen selbst ergriffen hat, so gibt es auch in der Seele solche und sogar bessere Maßnahmen, die ergriffen wurden, damit jeder, der will, sich vor einer trügerischen Lehre, vor einer heuchlerischen Wahrheit schützt, die die Form, aber nicht das Wesen der Wahrheit hat. 

In der Welt gibt es viele Lehren, aber nur eine von ihnen ist wesentlich und bildet Nahrung für die Seele. Die anderen sind vollkommen falsch oder bilden nur eine Art Vergnügen für den Verstand, indem sie ihm keine Zeit geben, sich um sich selbst und um das Wohl seiner Seele zu kümmern. Hier liegt die Gefahr für das menschliche Leben -- die Menschen denken, dass sie die Wahrheit haben und dass sie von ihr lernen, aber was sehen wir in Wirklichkeit? Ihr Leben ist schwach geworden, es wurde von Kopf bis Fuß mit Lastern angesteckt. Es ist eine offensichtliche Wahrheit, dass hier ein Trug beigemischt wurde. Andernfalls hätte dies nicht bemerkt werden können. Damit die Wunde heilt, muss nach passenden Medikamenten gegriffen werden. Diese Medikamente sind die Wahrheiten Gottes, die in Form von Geboten gegeben werden: Nicht dies oder jenes zu tun -- einfach keine böse Tat zu verrichten oder keinen schlechten Gedanken zu denken, denn in diesen Dingen liegt das Gift der Hölle. Und wenn geboten wurde, was nicht getan werden soll, dann sagt Gott, der seinerseits alles weiß, also allwissend ist, in positiven Geboten, was man gemäß seinem Willens tun soll. Er hat sehr kurz gesagt: Du sollst den Herr und dein Gott und deinen Nächsten lieben wie dich selbst. Und wenn du die Liebe als ein höchstes Gebot annimmst, hat sie an sich selbst diese Kraft, uns alle guten Wege zu eröffnen und uns zu zeigen, wo die höchste Tugend liegt, die der Kern des Lebens unseres Herzens ist. Und weißt du, warum ich zu dir über unser Leben spreche? Wenn die Liebe als Gesetz und höchste Tugend angenommen wird, wird sie für uns alle zur gemeinsamen Verbindung, und wir sind einander nicht mehr fremd, sondern nah.

Es soll dich nicht wundern -- siehe, diese Liebe deines Herrn hat mich gesandt, um zu dir zu sprechen und die Gedanken des Herrn mit deiner Seele auszutauschen, dich zu unterstützen, dich geistig zu erheben, damit du das siehst und schaust, was du bisher nicht gesehen hast, und begreifst, was du bisher nicht begriffen hast. Und verstehst du dabei, was für ein Geheimnis noch vor dir liegt? Diese Liebe, dieses mein Dasein, dieses mein Gespräch mit deiner Seele. Ich sehe und erkenne, dass du selbst noch zögerst, dass immer noch ein Zweifel in deine Seele eindringt -- ob dies eine Wahrheit sei oder nicht. Du fürchtest, ob Ich nicht doch ein Gespenst in deiner Seele oder in deiner Vorstellung sein könnte. Aber die Wahrheit, die Ich vor dich hinlege, ist diese: Prüfe und du wirst erkennen, dass Ich gütig und barmherzig bin; rufe mich an, und Ich werde dir antworten, und du wirst mich erkennen. Kann das Kind, das so viele Male Nahrung von seiner Mutter bekommen und so viele Male ihre Stimme gehört hat, an ihr zweifeln? Nein, sage ich, das ist unmöglich. Du jedoch, der du so viele Wohltaten von Gott bekommen und so viele Male Seine Stimme gehört hast, zweifelst noch an Seinem Dasein, und die Gedanken deines Herzens verführen dich. Und lass mich es dir klar und deutlich sagen, damit du weißt: Ich weiß wohl, dass deine Liebe zu Gott unvollkommen ist. Du sprichst über Ihn, aber du bist nicht bereit, alles Seinetwegen zu tun. Es ist dasjenige, was dich irritiert -- dein Zögern. Heute denkst du an eines, morgen an etwas anderes; heute fühlst du eines, morgen etwas anderes. Dein Glaube ist schwach. Wenn er so groß wäre wie das Senfkorn, so würde er Wunder wirken.

Erinnerst du dich an die Worte, die Ich an einer anderen Stelle im Wort Gottes gesagt habe, dass die vollkommene Liebe jede Sünde wegjagt? Dies habe Ich den Schülern Christi gesagt, dies sage ich auch euch. Ihr seid alle Schüler Christi, der Unterschied zwischen ihnen und euch ist nur ein Unterschied der Zeit. Hier sehe ich wieder, dass dein Herz beunruhigt wird, als ob ihr einen inneren Schmerz spürtet. Warum machst du dir Sorgen, was die Menschen denken? Ist es dir nicht wichtiger, was Gott von dir hält? Oder willst du noch immer unter der Last der menschlichen Überlieferungen und Irrtümer gehen? Täusche dich nicht, das Wort Gottes wird von menschlichen Auslegungen nicht verändert. Wer das Wort Gottes versteht, der soll es auch befolgen, wenn er nicht gerade ein Opfer tückischer Täuschungen geworden ist. Denn wer das Wort verstanden hat und es nicht befolgen will, weil er daran zweifelt, dass dieses Wort so anzuwenden und zu erfüllen sei, wird, wenn ihn sein Gewissen zu plagen beginnt, tausende von Auslegungen kreieren, die nichts weiter als Ausreden sind, um die Wahrheit zu vergewaltigen und sich selber zu beschwichtigen, dass ein Vers nicht so, sondern anders verstanden werden muss. Und warum -- nur damit er sich vor der Wahrheit rettet, die ihn dazu veranlasst, dies so zu handeln, wie er es vom Geist Gottes vernommen hat. Sieh' die Bosheit der Kirche, ihre Auslegungen und Verdrehungen des Wortes Gottes. Als Jesus auf die Erde kam, sprach er die Worte Gottes so klar aus, dass derjenige, der sie hörte, nicht sagen konnte, er habe sie nicht verstanden. Und gerade deswegen kam er vom Himmel, vom Schoß Gottes -- Er, der Herr selbst, der selbst die Wahrheit war, -- um über die Wahrheit zu sprechen und Zeugnis von Ihr abzulegen, von der Wahrheit, die Er selbst war, der verborgene und herzinnige Gott des Friedens. Was für ein besseres Zeugnis kann dir oder jemand anderem gegeben werden als das, wenn die Sonne des Lebens selbst kommt. Was für ein besseres Zeugnis kann jemandem, der nicht glaubt, gegeben werden? Sind das Licht und die Wärme, die jeder empfindet und fühlt, nicht das beste Zeugnis für die Wahrheit? Zweifellos. Wenn jemand das Licht Gottes und Seine Liebe in seiner Seele zu spüren beginnt, ist das dann nicht ein Zeugnis dafür, dass Gott diese Seele besucht und Seine Gegenwart auf eine unmittelbare und innere Weise offenbart hat? Ist dann noch Raum für Zweifel und Zögerlichkeit im Hinblick auf das Dasein Gottes? Nein.

Ich, der ich gegenwärtig bin und zu dir jetzt spreche, bin ich nicht das beste Zeugnis, das dir Gott persönlich gibt? Ich weiß, du hast das Wort Gottes mehrmals gelesen und gesagt: "`Gut, was hier gesagt wird, ist die Wahrheit; doch ist darüber schon vor tausenden von Jahren von Menschen, die mir in großem Maße ähnlich sind, gesprochen und geschrieben worden -- wer weiß."' Der Zweifel dringt erneut in dein Herz ein, und du sagst dir insgeheim, damit dich niemand hört: Ist dies nicht vielmehr menschliche Dichtung, die Gott zugeschrieben wurde? O, mein Freund, warum täuschst du dich und neigst zum Geist des Unglaubens? Das ist eine heimliche, eine sehr heimliche Sünde, die niemand sieht. Du denkst, dass du nichts Schlimmes damit begangen, niemandem eine Boshaftigkeit getan hast, indem du es zugelassen hast, dass dieser Gedanke sich einnistet und zum Hindernis in deinem Leben wird. Du verbitterst Gott, wenn du Ihn ständig mit diesen heimlichen Gedanken aus deiner Seele verjagst. Halte ein -- Ich sage dir, tu das nicht mehr! 

Es ist besser zu glauben, als nicht zu glauben. Erscheint dir das sonderbar? Ist es nicht angenehmer zu sehen, als die Augen zu verschließen? Ist es nicht gefälliger zu hören, als die Ohren zu verstopfen? Hast du nicht mit deinem kleinsten Zweifel, wie klein auch immer er sei, deine Herzensaugen auf Gott hin geschlossen? Und du sagst: "`Er kann gar nicht hier sein."' Du verstopfst deine Ohren mit deinem Zweifel und sagst zu dir: "`Im Stillen spricht Er nicht, ich höre Ihn nicht, und ich kann mich nicht von seiner Existenz überzeugen."' Sieh', das ist "`die Weisheit"'. Du hast als Mensch deine Augen geschlossen und sagst zu dir: "`Ich sehe niemanden."' Du hast deine Ohren verstopft und sagst wieder: "`Ich höre niemanden."' Und [für dich] sieht das so aus, als ob du dadurch eine wichtige Frage löst. Aber beweist du damit etwas, indem du so handelst? Nichts Positives. Du vollziehst nur eine arglistige Täuschung dir gegenüber und gegenüber deiner Seele aus inneren, schlimmen Überzeugungen. Du kannst deine Augen schließen und deine Ohren verstopfen, wenn du auf eine Grube zugehst und denkst, dass sie nicht existiere, aber wenn du in sie hineinfällst, wirst du schließlich gezwungen, willentlich oder unwillentlich deine Augen zu öffnen und deine Ohren freizumachen, um zu sehen und zu hören, wo du bist, wenn du diesen innerlichen, unheilbaren Schmerz deines Verfalls empfindest. Ich spreche zu dir wie zu einem Freund, unabhängig davon, was du tust und wie du handelst; Gott ist nah und sieht, was du tust. Du kannst vor Ihm nichts verbergen. Achte folglich darauf: Ich spreche zu dir; es ist eine unbestreitbare Wahrheit für dich, dass Ich allgegenwärtig bin und komme, um zu wiederholen, was ich vielen anderen auch vorher schon gesagt habe: "`Wer mir nachfolgt, wird nicht in der Dunkelheit wandeln."' Weißt du, wo sich diese Worte finden? Ich bin gekommen, um dich zu lehren, sie zu verstehen und in ihnen zu gehen. Weißt du, dass die Finsternis alles verdirbt? Sie ist der gefährliche Raum des Lebens. Dass jemand in der Finsternis wandelt, bedeutet, alles verloren zu haben, was die schöne Welt Gottes hat. Dass jemand im Licht wandelt, bedeutet, alles erworben zu haben, was diese Welt besitzt. 

Verstehst du mich, verstehst du das, was ich zu dir spreche? Ich weiß, dass du mich verstehst. Wenn du zu denen gehörtest, die nicht verstehen wollen, käme Ich nicht, um zu dir zu sprechen. Aber du willst verstehen, und deswegen bin ich zu sprechen gekommen, damit du mich verstehst. Und ich weiß, dass du das verstehen wirst. Richte, folglich, dein Gesicht zu Gott und bete zu Ihm, und du wirst einen inneren Segen der Erkenntnis empfangen. Denn es steht geschrieben: "`Diejenigen, die Gott lehrt, brauchen nicht von anderen gelehrt zu werden, die Wahrheit zu erkennen."' Aber Gott wird sie jeden Tag durch Sein Gesetz lehren. Er hat tausende von Wegen, auf welchen Er Seine Wahrheit weitergeben kann. Öffne dein Herz und gib Gott Raum, damit er eintreten und ganz in deiner Seele sein kann. Es ist schon genug Zeit in Nachlässigkeit, Zweifel, Schwanken, Zögern und tausenden von anderen Ausreden vergangen. Jetzt ist die Zeit günstig, ein auserwählter und dem Herrn gewidmeter Tag. Ein Tag, an dem du Ihm so dienen sollst, wie es Ihm wohlgefällt. Das werde Ich dich lehren und dir einen breiten Weg eröffnen -- zu gehen und den Willen Gottes zu erfüllen. Ich sehe, diese Worte beunruhigen dich sehr und du fürchtest dich sehr vor Bedeutungen. Halte dein Gewissen rein und lasse dich von nichts stören. Die Erde gehört Gott, und alles, was auf ihr ist. Somit kann derjenige, den Gott nicht verurteilt, von keinem anderen verurteilt werden. Ich spreche heute zu dir, und du sollst das wissen und in deinem Herzen behalten. Die Macht ist von Gott, und sie zeigt sich in deiner Kraftlosigkeit. 

Siehe, ich bin der, der das Wort Gottes in jedem Herzen, das es annimmt, belebt -- der Wahrhafte, der zu dir spricht -- und der dir die Geheimnisse des Reiches Gottes diktiert, damit du sie begreifst. Ich folge dir an jeden Ort und leite dich und bewahre dich in allem und gebe dir meinen Segen zu wachsen und dich bei der Erkenntnis meines Wortes zu vervollkommnen. Und ich will, dass du Diesem, der immer Licht und Güte ist, gleichen wirst. Dann wird sich Gott zeigen und Wohnung in euch nehmen und immer in euch leben. Himmel und Erde werden vergehen, aber meine Worte werden nicht vergehen, weil ich der lebendige Gott bin. Mein Wort wird immer bestehen bleiben, und ich belebe es von Geschlecht zu Geschlecht, damit mich alle kennen und fürchten. "`Ich habe dich weder getäuscht noch belogen, spricht der Herr, aber geprüft, um zu erkennen, ob du treu bist, und ob du meine Ratschläge befolgst, und ob du meine Gebote, die ich dir gebe, erfüllst."'

Geh auf meinem Weg und fürchte dich nicht -- befolge meine Worte und du wirst erkennen, dass ich sanft und barmherzig bin, und du wirst meine Treue erfahren, [nämlich] dass sie unveränderlich ist. Habe Geduld in deiner Seele und warte auf mich, bis ich mich in meine Kraft kleide und im Namen meines Rechts zu urteilen beginne. Hier, ich sage dir, dass ich jedem nach seinen Taten Recht sprechen werde, ich werde auf meinem Weg, auf dem du auf mich wartest, nicht zögern; ich werde mich offenbaren und dir zur richtigen Zeit helfen. 



1. Juli 1900




\section{Viertes Gespräch -- Leben und Wiedergeburt}

 An einer Stelle in der Heiligen Schrift wurde geschrieben: "`Wenn jemand nicht von neuem geboren wird, kann er das Reich Gottes nicht sehen. Was aus dem Fleisch geboren ist, ist Fleisch, und was aus dem Geist geboren ist, ist Geist."' Der aus Fleisch und Blut Geborene ist sterblich; in seinem Herzen wohnen vergängliche und von jeder Begierde und Wollust begleitete Dinge. Er kann in das Reich Gottes nicht eintreten, denn Fleisch und Blut können das Reich Gottes nicht vererben. Wer von Gott nicht geboren wird, kann weder das geistige Leben empfangen noch es verstehen, denn es wird geistig geprüft und empfangen.
 Siehe, die Auszeichnung Gottes, die Er jeder Seele anbietet, bevor sie sich in das Gewand der Unsterblichkeit kleidet, das ewige Leben zu empfangen, das selbst der Herr, der Erlöser ist. Siehe, dieses notwendige Bedürfnis für dich, von Gott und von Seinem Geist geboren zu werden. Deshalb bin ich gegenwärtig, um dieses Werk in deiner Seele zu vollbringen, das nach der Vorsehung Gottes, der gemäß Seinem ewigen Ermessen beschloss, dieses Gute zu tun, dich im Leben der Unsterblichkeit zu kleiden und dich anzunehmen in Seinem Reich als Sohn der Wahrheit, den Er bestimmt hat, Ihm zu dienen. Dieses Dienen ist ein Dienen gemäß der Wirkung des Geistes, der zu dir durch mich spricht und allein in deinem Herzen wirkt, um jene Wirkung der göttlichen Offenbarung in dir hervorzubringen. Und dieser Geist des ganzen Lichtes wird deine Erlösung vollbringen und bewirken und dir den Weg in das Reich Gottes eröffnen, damit du eintrittst und die herzinnigen Gaben des Herrn empfängst, die Er für dich bestimmt hat. Diese erwarten dein Dasein, damit du sie empfängst. Denn nur derjenige, der vollkommen von Gott und Seinem Geist geboren ist, kann die Geheimnisse des Reiches Gottes empfangen und nur er kann jeden Segen und jede Fülle des Geistes empfangen und eins mit dem Herrn werden. Ohne diese innere Verwandlung und innere Verneinung kann Gott niemals und auf keine andere Weise mit einer gefallenen Seele in Verbindung treten. Das ist Sein Weg, den Er selbst vor der Ewigkeit zugrundegelegt hat und durch den Er Seine Treue gegenüber Seinen Kindern einhält. Deswegen wurde gesagt, dass jeder, der von Gott und von dem Geist geboren wurde, Seine Stimme hört und zum Licht kommt, denn Gott ist Licht. Und jeder, der zum Licht kommt, empfängt Gott in seiner Seele; und jeder, der Seine Stimme hört, empfängt den Geist in seinem Herzen, damit die Wahrheit versiegelt und bewahrt wird. 
 
 Siehe, deshalb bin Ich gegenwärtig, um Zeuge des arbeitenden Segens Gottes und Vermittler des Geistes zu sein, der dich weiht und das vom Herrn, Gott jedes Lichtes Bestimmte vollzieht, dem Ruhm und Ehre gebühren. Folglich, möge deine Seele Frieden und stille Zuversicht haben, denn in den göttlichen Beschlüssen gibt es keinen Widerruf. Ich sage dir das, der Ich dich von Beginn an gelenkt und innerlich mit unermüdlicher Kraft veranlasst habe, mein Wort zu lesen, dich zu bemühen, in meine Fußstapfen zu treten, zu beten und nach dem höchsten Segen des Herrn zu trachten, damit du dich ständig in Ihm festigst und erneuerst. Und Ich bin es jetzt, der sich für dich einsetzt und dir wünscht, in der Erkenntnis der Wahrheit, die dich frei machen wird, zu wachsen. Ich warne dich von nun an: Hüte dich vor jeder willentlichen oder unwillentlichen, offensichtlichen oder geheimen Sünde. Die Sünde kann von nichts entschuldigt und von niemandem berichtigt werden außer von dem einen Gott und Herrn. Halte die Sünde, diesen Verrat des Teufels, diese seine Schöpfung, beiseite und hüte deine Seele mit dem ganzen Schutz, den der Geist dir erweisen kann. Und jetzt, selbst der Herr des Friedens bewahrt dich in Seinem Namen vor aller List des Teufels. Diese sind die Worte des Lebens, die ich dir wiederhole und die ich in meinem Geist für dich trage. Kämpfe vor meinem Angesicht mit jenem guten, ehrenhaften Glauben und jener heiligen Liebe, die Gott selbst mit der Wirkung Seines Geistes gebiert. Und sei Schöpfer des Wortes, indem du alles tust, was dem Einen Gott gebührt und wohlgefällt. Siehe, jeder arge Mund wird verstummen, jede lästernde Zunge der Hölle wird aufhören und der Herr, der Gott des Friedens selbst wird herrschen.
 
 Siehe, das ist mein Zeugnis, damit du weißt, was ich von der Wahrheit gezeugt habe. Und wenn du vom Geist der Wahrheit ergriffen wirst, mögest du wissen, dass Ich, dein Herr, der tot war und jetzt lebendig ist und in deiner Seele wirkt, das gesprochen habe. Und in Meinen Händen sind die Schlüssel der Hölle -- Ich öffne und schließe. Und ich weiß von allem, was unter dem Himmel getan wird -- es gibt keine feindliche Kraft, die ihre Taten vor meinem Auge verheimlichen kann. Siehe, ich teile dir mit, damit du weißt, dass es nicht der Wille des Teufels ist, der erfüllt wird, sondern es ist mein Wille. Und die Schafe, die ich habe, kann niemand aus Meinen Händen reißen. Möge sich dein Herz nicht fürchten. Tue alles, was dir Mein Heiliger Geist sagt. Die göttlichen Angelegenheiten und die göttlichen Worte werden im Leben durch Taten geprüft. Nicht dem, der will und selbst danach trachtet, gehört alles, sondern dem, dem Gott freiwillig von Seiner Gabe gegeben hat. Diese sind Meine Worte, die Ich dir sage. Präge sie in dein Herz ein und wisse, dass Ich nah und schon bereit bin, dir zu helfen. Ich werde all deine Wege glätten und schlichten. Du wirst von nun an und immer in Meiner Gegenwart gehen und Ich werde in dir jeden Tag ruhen, und all deine Taten werden wie Bäume zwischen Wasserquellen wachsen.
 Siehe die Weisheit Gottes, siehe die Werke Gottes, die Seinen Ruhm verkünden. Das, was du bald sehen wirst, wird dich begreifen lassen, was am Himmel für dich verborgen ist. Dort, in diesem Leben oben, wirst du die Wohltaten Gottes verstehen und Seine unausgesprochene Gnade begreifen. Dort, in diesem gemeinsamen Fest aller Söhne und Töchter Gottes, wirst du Meinen Ruhm sehen, den Ich vor der Erschaffung der Welt hatte. Dort, in Meiner Gegenwart, wirst du dich wie all meine Auserwählten freuen. Diese ist die Stimme deines Gottes und Herrn, der dir gegenüber so nachsichtig ist, dass Er Sein Leben in ein lebendes Opfer demjenigen gibt, der am Thron sitzt, damit du Ihn Selbst, der dich mit Fäden der Liebe angezogen hat, erkennst. Er ist derjenige, der Eine und wahre Gott des Lebens, der zu dir durch die Güte Seines Geistes spricht. Und alle Seine Worte sind lebendige Nahrung. Sie sind Worte des Lebens und wer sie hört und vernimmt, wird von Gott gelehrt. Er ist der Wahre, der von Anfang an spricht. Er ist der Wahre, der von oben herab- und hinaufgestiegen ist, der die Welt mit Seinem Ruhm erfüllt, der das Spiegelbild Seiner Macht in dem Vater des Lichts ist, in dem die ganze göttliche Fülle wohnt.
 
 All diese Körnchen und guten Samen, die ich heute in deine Seele gelegt habe, werden eines Tages wachsen. Und dann wirst du begreifen und meine Worte durch meinen Geist selbst prüfen, dass sie wahr und heilig sind, und dass Ich zu dir wirklich mit Worten der Liebe gesprochen habe. Du wirst dann den Geist deines guten Gottes ganz erkennen, und obwohl du Ihn mit deinen Augen noch nicht siehst, wird die Zeit kommen, wenn sich deine Augen öffnen werden und du mich mit deinem ganzen Herzen, mit deinem ganzen Verstand, mit deiner ganzen Kraft und Seele erkennen wirst. Dieser wird er, dein Tag, wenn du all meiner Segensprüche empfangen. 

 All das erscheint dir wunderlich, all das lässt dich staunen. Aber sei treu und du wirst erkennen, dass dies meine Worte sind. Die Wahrheit ist der Geist des Wortes. Der Geist ist es, der Leben gibt, der Buchstabe nutzt uns nicht. Das, was ich zu dir spreche, ist Geist und Wahrheit. Die Wahrheit bin Ich, dein Herr und das Leben bin Ich, dein Gott, und mein Geist ist dein Leiter, der dich lenkt, der dich lehrt und der mein Wort jeden Tag in deinem Herzen belebt, damit du mich erkennst. Du hast jeden Tag mein Wort gehört. Siehe, ich habe ich dich früh geweckt und zu dir wie zu einem Freund gesprochen habe. Ich habe mein Wort vor dir gezeigt und dich die Wahrheiten des Reiches Gottes gelehrt. Ich habe dich innerlich überzeugt, indem ich vor deinen Augen offenbart habe, dass du meine Taten siehst und meine Wahrheiten im Leben erkennst. Ich habe dir klar gesagt, wie du in jedem Fall handeln sollst. Du hast immer meine Stimme gehört und erkannt. Je näher du mir gewesen warst, desto klarer und zur Vernunft bringender war sie für deine Seele. Es gab auch Minuten in deinem Leben, als du dich geirrt hast, aber du weißt und es gibt einen Zeugen dafür, dass ich dich niemals habe fallen lassen. Ich bin immer nah gewesen -- immer das Verwirrte zu berichtigen und alles umzukehren, damit es dich r im beim Guten unterstützt.
 
 5. Juli 1900






\section{Fünftes Gespräch -- Erhebung, Geist und Seele}

Seitdem ich begonnen habe, zu dir zu sprechen, sehe ich, dass es viele Dinge gibt, die deine Seele beunruhigen. Dein Herz durchläuft einen Übergangszustand. Dein Verstand scheint angespannt zu sein. Die Gedanken und die Gefühle, die deine Seele bewegen, haben eine zeitweilige Finsternis im Urteil deines Verstandes hervorgerufen. Die Widersprüche, die ununterbrochen in deiner Seele aufeinander stoßen, haben schmerzhafte Gefühle ausgelöst. Du fühlst dich so, als ob du nicht Herr deiner selbst wärst. Hier entsteht jedoch ein Kampf zwischen den niederen und den höheren Gefühlen deiner Seele. Du bist zwischen zwei Lagern von wirkenden Kräften -- den Kräften des Guten und des Bösen -- gesetzt worden, die darum wetteifern, wer den Vorrang im Staat deines Geistes haben soll. Einerseits zieht dich die Welt mit ihren Verlockungen und Schmeicheleien an, indem sie dir ihre Ansprüche auferlegt, [nämlich] es sei vorteilhafter, so zu leben wie alle anderen auch. Und sie warnt dich, dass du mit jeder Abweichung von ihren Normen für einen unzeitgemäßen Menschen, für dumm und unvernünftig gehalten wirst, und keinen Nutzen aus deinem Leben ziehen kannst. Dein Gewissen, das innere Selbstbewusstsein deines Herzens, von den hohen Impulsen der Liebe und des Guten ergriffen, ruft dich dazu auf, deine Pflicht zu erfüllen. Ist das nicht die Stimme deines Herrn, der dich ruft, dass du dich daran machst, Seine Arbeit zu tun? Ja. Und du sollst wissen, dass Er groß und ruhmvoll ist. Die Grenzen Seines Reiches sind unermesslich. Es ist nicht nur diese Welt, die ein Korn inmitten des Ozeans Seines unermesslichen Machtbereiches ist. Können dich und deinen mächtigen Geist, der du erstrebst und wünschst, deine Flügel auszubreiten und gen Himmel zu fliegen, vergängliche Dinge und Nutzen zufriedenstellen? Nein, sage ich, sie sind eine Täuschung. In ihnen gibt es keinerlei Nahrung, nichts, wonach du suchst. Sie haben nur augenscheinlich gute Gestalt und Edelmut. Aber kannst du zumindest einen sehen, der sie angenommen hat und besser geworden ist oder aber sich dem Himmel genähert hat? Nein, keine Dinge, in die die Sünde eingedrungen ist, können die himmlische Wohnstätte betreten. Nein, keine Seele kann den Himmel betreten, bis sie sich von jeder Sünde, von jedem Fleck, wie klein auch immer er sei, gereinigt hat. Das ist eine Wahrheit, die Gott selbst in früheren Zeiten ausgesprochen hat. Weißt du, wo jene ruhmvollen Wesen sind, die einst die Vorräume des Himmels füllten und nur wegen einer Sünde des Ungehorsams, nur wegen der einen Last der Lästerung, nur wegen eines Flecks Unreinheit zum ewigen Exil aus der himmlischen Wohnstätte verbannt wurden? Im Himmel gibt es kein Ansehen der Person, es gibt keine Bevorzugung des einen gegenüber dem anderen. Alle, kleine und große, leben in unaufhörlichen Verbindungen mit der Liebe. Dort sind alle Söhne, dort sind alle Priester, dort sind alle Könige und Diener Gottes -- das auserwählte Geschlecht, die königliche Nachkommenschaft.

Die Prüfung ist jedoch hier unten, wo jeder, der geboren wird, dazu verpflichtet ist, alle Prüfungen des Lebens zu durchlaufen, um sich dem Weg der Vollkommenheit, dem Weg des Heiligen zu nähern. "`Ihr sollt heilig sein; denn ich, der Herr, euer Gott, bin heilig."'\footnote{3.Mose 19,2.} Dies sind die wichtigen Dinge im Leben -- die Erkenntnis der Wahrheit, die einen höheren Wert als alles andere hat. Die Wahrheit ist Gott selbst, und wer Ihn annimmt, wird frei und selig. Gott selbst ist lebendiges Wasser und wahres Brot, und wer dieses Brot annimmt, wird lebendig sein, wie Er auch lebendig ist. In deinem Leben soll, wie ich vorher zu dir gesagt habe, jene große Verwandlung geschehen, über die ich in meinem vorigen Gespräch gesprochen habe. Ohne diese innere Wandlung ist es unmöglich, nach oben fortzuschreiten. Ohne sie wirst du einem Blinden gleichen, der die Schönheiten der göttlichen Welt sehen will. Ohne sie gleichst du einem Toren, der die Wege der Weisheit und die Verordnungen des Lichts begreifen will. Dies ist die Hauptbedingung: Dich von jeder Sünde zu befreien. 

Es ist von Gott gesagt worden, dass jeder, der aus Gott geboren ist, keine Sünde begeht, denn der Geist Gottes wohnt in ihm. Innerlich fragst du dich jedoch, wie jemand erkennen kann, wann er aus Gott geboren ist. Ich antworte dir: Wenn er in sich selbst die Merkmale, die für Gott charakteristisch sind, sieht; wenn in ihm die Liebe, die Wahrheit, die Tugend in ihrer Fülle wohnen; wenn es in ihm Frieden und Eintracht in jedem Gedanken und Gefühl gibt, wenn die Widersprüche des Lebens aufhören, seinen Verstand zu stören; wenn die Unzufriedenheit sein Herzen verlässt; wenn die Boshaftigkeit und die Lüsternheit aufhören, Schatten auf sein geistiges Leben zu werfen, und er mit einem neuen Selbstbewusstsein, wie ein neugeborener, neuerretteter Mensch sich selbst in einer anderen Welt, die ganz anderer Natur ist, stehen sieht und er von einer ganz anderen Art von Dingen und Gedanken bewegt wird; wenn die Sanftmut, die Barmherzigkeit, die Offenherzigkeit, die gute Absicht, das Mitleid und die völlige innere Selbstvergessenheit, sein Leben als lebendiges, heiliges und gottgefälliges Opfer für das Gute und für den Ruhm seines Werks darzubringen, ohne nach seinem eigenen Willen und seinen eigenen Angelegenheiten zu suchen. Siehe, das bedeutet es, aus Gott geboren und entsprechend deinem Leben Ihm gleich zu sein. Das ist eine Bedingung, ein großes Bedürfnis für eine Seele wie die deinige, die überall nach Gott sucht. Du weißt wohl aus eigener Lebenserfahrung, dass jeder Same die Eigenschaften seiner Art besitzen soll, und du weißt wohl, dass nur durch diese Eigenschaften unterschieden werden kann, zu welcher Art die Samen gehören. Hier ist eine große Wahrheit, die die Menschen, die Tiere, die Bäume unterscheidet. Aber diese Wahrheit selbst unterscheidet auch den Menschen in sich selbst. Im Herzen und in der Seele des vom Fleisch geborenen Menschen, in dem Fleisch und Blut herrschen, können sich die Eigenschaften, Gefühle und Gedanken eines von Gott geborenen Menschen, in dem der Geist die Herrschaft hat, nicht manifestieren. Denn beiden Seelen haben einander entgegengesetzte, miteinander unvereinbare Naturen. Wo Fleisch und Blut herrschen, kann weder der Geist seine wohltuenden Taten manifestieren noch das Bewusstsein solch einer Seele die geistigen Dinge einer anderen Seele, die in einem ganz anderen Umfeld, in einer viel höheren Region als sie selbst lebt, wahrnehmen und begreifen. Und deswegen wurde gesagt, dass der natürliche Mensch die Dinge nicht verstehen kann, die vom Geist sind, denn sie lassen sich nur geistig verstehen. Und das ist richtig und wahr. Es ist richtig, denn es ist wahr; die Worte Gottes werden ununterbrochen täglich bestätigt.

Alle, die sich Gott nähern, müssen sich grundsätzlich einer inneren Wandlung unterziehen. Das ist ein allgemeines und unerbittliches Gesetz des Lebens. Im seinem ganzen Reich ist ebendiese Regel zu bemerken. Alle Wesen, von den kleinsten bis zu den größten, unterliegen derselben inneren Wandlung. Keine lebendige Seele kann einen Schritt höher hinaufsteigen, wenn sie sich nicht innerlich wandelt und sich auf ein höheres Umfeld, auf eine höhere Ebene, die sie anstrebt, vorbereitet. Das ist die allgemeine Ordnung in diesem vergänglichen Leben, in welchem die Wandlung und die Artveränderung ihm als beginnendes Leben eigen sind. Die Vervollkommnung, das innere Selbstbewusstsein, die Erhebung des seelischen Lebens, die Reinheit des Herzens, der Edelmut des Verstands, die Würde der Seele, das Licht des Willens sind jedoch Eigentum und Vorteil nur des geistigen Menschen, der nicht vom verdorbenen Samen des menschlichen Sohnes, sondern von Gott, von einem heiligen Samen geboren wurde. 

Die Dinge, die Gott erschaffen und geordnet hat, sollen sich immer an ihre Ordnung halten. Erstens das Natürliche, dann das Geistige; erstens das Sichtbare, dann das Unsichtbare; wie in einer Schule, in der die Bildung mit einer Gegenstandslehre beginnt und erst dann zum reinen Denken des Verstandes fortschreitet. Zuerst sollen das Auge, das Ohr und jedes andere Gefühl affiziert und durch Tasten und Fühlen geübt werden und dann sollte zum inneren Begreifen der Dinge übergegangen werden. Der Vergleich ist hier angebracht. Auf solch einfache und sensuelle Weise soll jede unentwickelte Seele, in der das Natürliche einen Vorsprung vor dem Geistigen hat, den Anfang machen -- denn in jedem Anfang gibt es solche Bedingungen. Aber der für den Himmel Erschaffene soll eines Tages unumgänglich, außerhalb jeden Hindernisses aus dem Bereich des Natürlichen und Vergänglichen austreten und in die Grenzen des Ewigen eintreten; denn nur hier befinden sich alle Bedürfnisse für die Vervollkommnung und die absolute Vollkommenheit der Seele. Sie soll dorthin zurückkehren, wo sie entstanden ist. 

"`Und Gott hauchte Adam von Seinem Atem ein, und so wurde er eine lebende Seele"'. Der zweite Adam jedoch, der aus dem Himmel stammt, ist Gott selbst; der wurde zum lebendig machenden Geist. Wie es für den ersten so leicht und natürlich war, zu sündigen und sich nicht an das Versprechen und den Befehl zu halten, so war es für den zweiten umgekehrt, der selbst die Macht hatte, durch Seinen Geist das wiederherzustellen, was verloren war, und sich mit der Einheit des Lebens zu versöhnen. Denn im Anfang, als Gott Adam schuf, schuf Er ihn allein, indem Er ihn vor jedem Fall und Sturz bewahrte, damit das Böse keinen Platz hatte, sich in seiner Seele einzunisten. Da er aber aus eigener Erfahrung herausfand, dass sein gegenwärtiger Zustand ihm unangemessen war und in ihm keine Annehmlichkeit hervorrief, verlangte er von Gott einen ihm gleichenden und angemessenen Gefährten. All dies zeigte, dass er noch nicht imstande war, Gottes Umgang mit ihm wertzuschätzen. Er war noch nicht geistig genug, um die geistigen Ordnungen zu begreifen -- er war eine lebendige Seele, aber kein lebendiger Geist. Er zog die seelischen Gefühle und sich daraus ergebenden Vorteile den geistigen vor. Er sah in der natürlichen Welt, dass die niederen Geschöpfe zu zweit gehen und dass es eine gewisse Verbindung zwischen ihnen gab. Und ihm schien, dass es in diesem verborgenen Umgang eine gewisse Annehmlichkeit gab. Er verlangte gemäß diesem Impuls von Gott, auch einen ähnlichen Gefährten zu haben, sah jedoch nicht, dass sich durch diesen Gefährten die Tür seines Falls öffnen würde. Dieser Gefährte, welcher die Frau war, sollte alle Leidenschaften wecken, die in seiner Seele schliefen. Und Gott wusste, dass Adam nicht imstande sein würde, Herr über sich selbst zu werden, und dass er sich selbst zum Leiden verdammen würde. Denn der Baum der Erkenntnis von Gut und Böse, von welchem zu kosten ihm verboten wurde, war die Frau. Die Versuchung stand schon vor ihm, und Gott musste ihm mit einem Gebot verbieten, davon zu kosten, denn an dem Tag, an dem er vom Baum der Erkenntnis kosten würde, würde er sterben. Aber konnte er vor einem augenscheinlich so angenehm anzusehenden Baum stehen bleiben, ohne seine Früchte zu kosten? Nein, das war gemäß Adams Natur unmöglich. Und endlich überzeugte ihn die Frau, von dieser Frucht zu kosten, indem sie ihm mit einem Beispiel voranging und ihm einen verlockenden Preis anbot, nämlich dass er Gott ähnlich werden würde, alle Dinge zu kennen, dass er die Kraft haben würde, seine eigene Nachkommenschaft zu zeugen und sich an ihr zu erfreuen und alle Ehren zu erhalten. "`Das ist besser"', bestätigte sie, "`als dass wir nur zu zweit sind und uns für immer in diesem Garten herumtreiben, nämlich dass wir Herrscher über die ganze Erde und ihre Reichtümer werden."' Gott sah die Zukunft voraus, und als Er Sein Gebot erlassen hatte, um Adam zu prüfen, sagte Er zu Adam: "`Seid von nun an fruchtbar und vermehrt euch und füllt die Erde"'\footnote{1 Mose 1,28.}

Aber konnte er die Erde füllen und sie beherrschen, ohne das Gebot zu übertreten, ohne die Frucht der Erkenntnis von Gut und Böse zu kosten und ohne diesen inneren Sinn des Lebens zu begreifen? Nein, das war unmöglich innerhalb der Grenzen der Dinge und des Lebens selbst, das er angenommen hatte. In ihm lebte bereits eine lebendige, erwachte, mit Urteil begabte Seele, die alles kosten und erfahren wollte, unabhängig davon, ob gut oder böse -- das gefährdete ihn nicht, insofern er seinen Wunsch erfüllte. Siehe, der unvermeidbare Moment des Fallens und des unglücklichen Weges, durch den sich das Böse eingeschlichen und mit all seinen Folgen und Gräueln ins Leben eingetreten ist. Das Gesicht der Erde solle durch das Feuer der Hölle in Brand gesetzt und von all ihren Sünden und Verbrechen gereinigt werden.

Aber da es Adam nicht möglich war, Gerechtigkeit zu üben, weil er im Fleisch war, sah Gott vor, selbst Fleisch zu werden und die leidende Nachkommenschaft Adams aus dem Land der Sklaverei hinauszuführen. Und deshalb sagt Gott, dass diejenigen, die für würdig gehalten werden, jener Welt und der Auferstehung von den Toten teilhaftig zu werden, nicht heiraten und auch nicht verheiratet werden, sondern sie sind Engeln gleich.\footnote{Vgl. Lk 20,35-36.}. Das Böse wird keinen Platz haben, denn alle, die durch Gottes Wort schon gerettet sind, indem sie alles erfahren und gekostet und aus tiefer innerer Erfahrung gelernt haben und die Folgen kennen, die aus jedem Wunsch und jeder Tat hervorgehen, werden die Last der Sünde und des Verbrechens, die sie umgarnt und erneut ihren Fall hervorrufen kann, ablehnen. Der geistige Mensch kann jedoch keine Sünde begehen, denn die Sünde selbst ist etwas, das ihm nicht eigen ist. Die Sünde und die Täuschung können keinen Boden in seiner Seele finden, um zu wachsen. Er ist frei von ihrem Einfluss und von ihrer Schlinge, wie Gott selbst frei ist. Denn wie der Geist Gottes an einer Stelle sagt, wird jeder von seiner eigenen Begierde und von seinem eigenen Wunsch getäuscht, die, wenn empfangen, die Sünde gebären, und die Sünde, wenn begangen, gebiert den Tod. Diese inneren Unfälle der Seele sind richtig. Aber du wirst fragen, aus welchem Grund. Siehe, warum: Jede Kraft oder jeder Segen, die vergeudet werden, ohne eine als Nahrung für die Seele würdige Frucht bilden zu können, sind ein irreparabler Verlust für sie; und es ist dir bekannt, dass jede Seele, die ausgibt, ohne zu verdienen, zu zeitweiligen oder ewigen Leiden verurteilt ist. Für den, der sie kennt, ist das eine Wahrheit, die keinen Widerspruch in sich trägt. Nicht jeder, der zu Mir "`Herr, Herr"' ruft, wird in das Reich Gottes eintreten, sondern der, der den Willen Meines Vaters erfüllt. Und denkst du, dass Seine Worte nicht bestätigt werden, wenn Gott spricht? Nein, Himmel und Erde werden vergehen, aber meine Worte werden nicht vergehen.\footnote{Vgl. Mt 13,31.}

Hüte dich folglich vor jeder Lehre, die die Wahrheit Gottes zugrunde richtet und bestreitet. Alles, was die Menschen sagen, wird vergehen und vergessen werden, die Wahrheit Gottes, deines Herrn, wird für immer, wie eine unerschütterliche Säule, wie ein Fundament stehen, auf dem der ganze Himmel gründet. 

Derjenige kann nicht vom Geist der Wahrheit geboren werden, der Sünden begeht, und in dem Trug und Begierde herrschen. Niemals darf ein solcher Mensch in das Reich Gottes eintreten. Niemals kann sich eine solche Seele entwickeln, zur Vollkommenheit gelangen und eine würdige Frucht bilden. Solch eine Seele, solch ein Mensch gleicht einer Ähre ohne Körner. Kann eine solche Ähre ihr Leben oder das Leben ihrer Gattung fortsetzen? Nein, auf keinen Fall. Siehe, deshalb wurde gesagt: Die Gottlosen aber werden getilgt aus dem Land, und die Treulosen reißt man aus ihm heraus.\footnote{Sp 2,12.} Siehe, darin wirkt Gott selbst, der Seine Gebote allein anwendet und erfüllt. Niemand wird Seiner Strafe entfliehen können. Er ist gerecht und heilig und richtet über jede Tat und jede Handlung und wird eines Tages Sein Urteil aussprechen. Wenn Seine Gnade groß ist, dann gleicht ihr auch Seine Gerechtigkeit. Wenn Seine Liebe groß ist, dann ist auch Seine Heiligkeit die gleiche. Gott ist der Eine, rein und heilig. Für Ihn haben die Unterschiede keine Bedeutung. In allem tut Er die Absichten Seines Geistes und Seines höchsten Willens. Wie Er sanft und nachsichtig gegenüber einer Reue zeigenden Seele ist, so streng, gerecht und heilig ist Er, einen alten, im Bösen verhärteten Sünder zu verurteilen. 

Aber diese Dinge sind nicht die wichtigsten für dich, zu wissen, auf welche Weise Gott sich erbarmen und verzeihen sowie verurteilen und vernichten kann. Wichtig ist deine Erlösung, deine Erneuerung und Vervollkommnung auf dem Weg Gottes. Wichtig ist deine Aufklärung, die Wahrheit zu kennen, die Gott selbst ist. Nach dieser inneren Erkenntnis sollst du nach ihr trachten und dich in sie kleiden. Dann wird dein Dienst notwendig Gott wohlgefallen, und das Werk deiner Hände wird angenehm vor Seinem Gesicht sein. Denn erfüllst du Seinen Willen so wie der Herr selbst, wirst du immer und zu jeder Zeit gehört werden. Und es wird nichts Unmögliches für dich geben. Du wirst rufen, und dir wird geantwortet werden, du wirst bitten und bekommen. Das ist der große Segen Gottes, vor Ihm immer wohlgefällig zu sein. Und ist das nicht das höchste Gut, immer tun zu können, was immer der Geist will? Ja, das nämlich ist der Dienst dem Geist gegenüber. 

Siehe, ich habe dich über einen Gegenstand aufgeklärt, der für dein geistiges Leben so wichtig und nützlich ist. Das, was ich dir gesagt habe, ist nicht für die Welt, sondern für dich selbst. Denn seit langer Zeit habe ich deine Schwierigkeiten und den ausweglosen Kreis deiner Bemühungen gesehen. Dein ständiger Fall und dein Wiederaufstehen haben Gnade und Mitleid in meinem Geist hervorgerufen. Da habe ich zu mir gesagt: "`Siehe, eine Seele, die nach Mir sucht, die in der Finsternis des Unwissens herumtappt, die ständig nach Licht und nach dem Ausgang aus dieser ausweglosen Situation dieses vergänglichen Lebens trachtet."' Da habe ich meine Hand wie ein Verteidiger, Vater, Bruder und Freund ausgestreckt und dich ergriffen, ohne dass du Mich erkannt hast. Und ich habe dich zum Ort der Erlösung geführt, damit du gerettet wirst. Ich habe Deinen Weg wohlgeordnet und alle Mittel vorbereitet, die für dich notwendig waren, damit du unter meiner Leitung lernen und dich erziehen kannst. 

Schritt für Schritt musste ich mit dir gehen, um dich davor zu bewahren, zu fallen und zu stürzen, und in allem, was deine Seele in dieser Welt zu erlangen wünschte, tat ich so, dir die Möglichkeit zu geben, alle diese Güter und bitteren Dinge zu erfahren. Ich habe dich mit allen Methoden des Wissens bewahrt und deinem Verstand Flug verliehen, auf dass er auf alle höchsten Orte, zu welchen dem Menschen der Zutritt gestattet ist, hinauf- und wieder herabsteige. Und dabei sehe ich jedoch mit Überraschung, dass dich das überhaupt nicht dankbar macht. Dein innerer Geist ist unruhig. Du trachtest nach dem, was weder in dieser Welt, noch in diesem Leben zu finden ist.

Heute ist der letzte Tag deines Kurses, und du wirst eine Prüfung vor Mir selbst ablegen. Und wenn du erfolgreich bestehst, werde ich dich in einen neuen Bereich, in ein neues, dir bis jetzt unbekanntes Leben einführen. Dort werde ich dich alles lehren und dir zeigen, wie du handeln und arbeiten sollst. Wenn du die Bedeutung und den Geist dieser Worte verstehst, dann hast du meinen Weg bereits eingeschlagen und dich Mir genähert. Das Werk des Glaubens wird jene innere Verwandlung hervorrufen. Denn ohne Glauben ist es unmöglich, Gott wohlzugefallen. Und so, wenn du die Wahrheit weißt, bleibt es noch übrig, dass du sie begreifst. Und wenn du sie begreifst, wird der Tag deiner Geburt in Gott kundgetan. Und wenn du neu geboren wirst und in die neuen Rahmen des Lebens Gottes eintrittst, wirst du erkennen, wie du erkannt worden bist. 


5. Juli 1900



\section{Sechstes Gespräch -- Der Weg und die Wahrheit}




Lege in dein Herz alles, was ich dir bis jetzt gesagt habe, weil die Zeit meine Worte rechtfertigen und die Wahrheit bestätigen wird, die ich zu dir von Herzen gesprochen habe, weil Ich heute und morgen der Wahre bin. Siehe, die innere Erneuerung, die sich in deiner Seele vollzieht, wirst du mit deinen eigenen Augen sehen. Ich werde deine Gedanken, die Kräfte deiner Seele wiederherstellen, dein Herz wird sich in Weisheit und Wissen kleiden, du wirst dich meiner Gegenwart erfreuen, und dein Leben wird vom Tod ins Leben übergehen. Der Unglauben jeder Seele ist das Haupthindernis für die Erlösung eines jeden von euch, und erstaunlich ist dieser Zustand, den ihr euch aus Unglauben und Trägheit selbst auferlegt habt. Wenn ich an euer Herz klopfe, ist die Tür viele Male geschlossen, und alles deutet darauf hin, dass die Tür an ihren Griffen verrostet ist. Wie viele Male habe ich euch, als ich vorbeiging, in eurem Leben nachlässig und geistig schlafend vorgefunden? Euer Körper und eure Seele waren munter, euer Geist und das Innere eures Herzens jedoch nicht. Und hier liegt ein Hauptgrund für die Verzögerung eurer Wiedergeburt. 

Es gibt keinen Zweifel, dass du nur dasjenige, was ich zu dir spreche, hörst. Dein Verstand war bis jetzt mit vielen Dingen, jedoch nicht mit der Wahrheit beschäftigt. Du hast viele Dinge, aber nichts Besonderes begehrt. Begehrt hast du alles, doch ist daraus nichts hervorgegangen. Wo sind deine Gedanken und Wünsche -- wohin sind sie jetzt verflogen, was ist mit ihnen geschehen? Was für einen Wandel siehst du heute in deinem Leben im Vergleich zu früher? Ist das nicht eine Täuschung für dich selbst? Ja, ohne jeden Zweifel. Sei jedoch Gott dankbar und verbunden, dass Er nicht zugelassen hat, dass du deine Seele verlierst. Und das ist das höchste Gute, das Er für dich tun kann. Wenn ein Mensch in dieser Welt seinen ganzen Reichtum verliert, sein Leben jedoch rettet, dann kann man sagen, er habe nichts verloren, sondern im Gegenteil gewonnen. Was für einen Nutzen würde es ihm bringen, wenn er sein Leben verlieren und Reichtum gewinnen würde? Keinen, es wäre unsinnig. Siehe, warum der Herr zum Reichen sagt: "`Tor, diese Nacht werde ich nach deiner Seele fragen; wem wirst du all dies hinterlassen, was du erworben hast?"' So wird es mit jedem geschehen, der nicht im Herrn sondern in der Welt reicher wird. Ein Reichtum ist die Seele, und wenn jemand seine Seele erlöst, wird er reich wie derjenige, der im Diesseits seinen Reichtum für die Erlösung seines Lebens opfert. Denn das Leben kann den Reichtum, den es verloren hat, erneut verdienen, der Reichtum selbst kann jedoch das Leben nicht verdienen. In der Sage vom gerechten Hiob gibt der Herr ein gutes und wunderbares Beispiel für dich. Begreife folglich den Inhalt dieser Worte und sei nicht undankbar, sondern dankbar. Deine Seele, die der Herr bewahrt und geliebt hat, ist die größte Gabe, die dir der Herr einst gegeben hat. Ist das nicht wahr: Was nutzt es dem Menschen, wenn er die Welt verdient und seine Seele verliert? Das ist ein furchtbares Übel, das einen Menschen ereilen kann, das schönste und kostbarste Geschöpf, seine Seele, zu verkaufen und zu vernichten. Ist das nicht der größte Unsinn, den ein Sünder gegen sich selbst tun kann? Zeigt das nicht die Spitze des Zerfalls, der höchsten Gesetzlosigkeit gegen Gott und gegen den Geist selbst -- dasjenige, was in Ihm selbst am heiligsten und herzinnigsten ist, zugrunde zu richten. Ist das nicht ein verurteilungswürdiges Verhalten und eine der Hölle werte Tat? 

Ja, siehe, das ist das unheilbare Übel des Lebens, das kein anderer, sondern nur der Mensch sich selbst zufügen kann. Niemand kann das Leben einer Seele zugrunde richten, außer der Mensch selbst. Sei dankbar, dass dich Gott selbst für die Erlösung vorherbestimmt hat, weshalb Er arbeitet und handelt, um Seine Absicht zu vollziehen. Er hat dich bestimmt -- das ist ein herzinniges Geheimnis, das der eine Herr allein weiß, nämlich warum Er es tut. Wer die Stimme des Herrn hört, soll auf Seine Worte hören, und bei diesem Hören vollzieht sich jene wohltuende Wirkung der göttlichen Erneuerung der Seele. Wie wohltuend die Frühlingssonne und der Frühlingsregen sich auf das Pflanzenleben auswirken, so wohltuend wirken sich auch die Ankunft des Geistes Gottes und Seine Stimme aus. Wie gut erklärt das der Geist des Herrn, wenn Er sagt: "`Diejenigen, die Seine Stimme hören, werden zum Leben erwachen."' Und du, der du meine Stimme hörst und vernimmst, bist lebendig und lebst ein Leben, von welchem du nicht weißt, woher es kommt. Dieses Leben selbst ist jedoch der Herr des Lebens. Sein Heiliger Geist belebt dich, indem Er zu dir spricht, denn Er ist im ständigen Gespräch mit deiner Seele, die Seinen Geist atmet und wahrnimmt.

Das ist ein großes und herzinniges Geheimnis des Reiches Gottes, welches gewusst und empfangen wird. Der Geist ist das Leben -- das sagt der Herr, und du sollst an Seine Worte glauben. Wer die Worte Gottes hört und sein Gesicht im Gebet zu Ihm richtet, dem wird alles klar. Das Wissen und die Weisheit gelangen unmittelbar in seine Seele wie das Licht. Und der Herr ergreift die Herrschaft, und das Leben gewinnt an Bedeutung. Was für eine Herrschaft ist die Herrschaft des Herrn? Sie gleicht der Herrschaft der Sonne über den Tag. Ist solch eine Herrschaft nicht angenehm, wenn das Herz, das den Tag erwartet, den Tagesanbruch und den Aufgang der Tagessonne sieht, das ganze Innere des Menschen mit Freude erfüllt wird? Jede Seele gerät geistig in solch einen Zustand, wenn der Herr herrscht. Das, was ich zu dir sage, ist begreiflich für dich. 

Wahres Licht ist der Herr. Diese Naturdinge, die du siehst, sind nur ein Zeichen, ein Symbol und eine Erklärung für die geistigen Dinge. Denn die sichtbare Welt wurde in ihren Hauptzügen als ein Abbild der geistigen Welt erschaffen -- die Ordnung wurde von oben entlehnt. Die Wesensnatur ist in ihrer Gesamtheit eine Verkörperung der unsichtbaren Welt. Die Natur drückt das Leben und die Handlungen aller Lebewesen und Geschöpfe aus, die Gott erschaffen hat. Diese Ordnung, diese große Bühne, diese Werke der sichtbaren Welt wurden erschaffen, um euch zu belehren. Für euch, die ihr bestimmt seid, den Himmel zu erben, hat Gott all dies erschaffen, um euch näher an Sich heranzuziehen. Und alles zusammen dient als Hilfe für das Höchste und das Vollkommenste. Es sind nicht die Zeichen, die Sätze und die Worte, die ihm das Angenehme und die Schönheit verleihen -- es ist der Inhalt, der ihnen innewohnt, der Geist, der weht. Alle anderen Dinge sind Mittel, Stufen, Hilfsmittel, Bequemlichkeiten, Erleichterungen. Jedes von ihnen ist zweckmäßiger als das andere, um den göttlichen Gedanken, der an euch gerichtet wird, zu übermitteln. Und wenn dieser göttliche Gedanke eindringt und sich an den intimen Haltepunkten der Seele niederlässt und seine Botschaft des Geistes überbringt, bildet er jene innere Verbindung, die Ausdruck der sichtbaren Liebe des Herrn ist. Und was für ein Auftrag ist das, den Sein göttlicher Bote eurer Seele bringt, außer dem Zeugnis, dass Seine Gnade und Seine Güte euch gegenüber nicht geringer geworden ist? Ist es nicht Er, der deinem Herzen beständig Seine Zuversicht sendet, dass die Unglücke, die euch ereilt haben, nicht irgendwelche Zeichen und Prophezeiungen sind, dass die Bänder des Bündnisses zwischen dir und Ihm gerissen sind und Gott Sein Antlitz gegen dich wie gegen einen Feind gewandt hat? Nein, im Gegenteil, Er kommt, um dir die Zuversicht zu geben, dass das Gewitter auf dieser Welt, das einen Ast deines Lebens gebrochen hat, dass der Regenschauer, der die Wurzeln deines Baums freigelegt hat, oder der Reif, der das eine oder andere Blatt hat erfrieren lassen, sich in das Gute verwandeln werden. "`An die Stelle des gebrochenen Astes"', sagt Er, "`wird ein viel besseres Ästchen aufgepfropft."' Die freigelegten Wurzeln wird Er mit einem viel besserem Boden bedecken und Er wird um sie herum einen besseren Boden aufschütten, den Er umzäunen und bewahren wird. Und an der Stelle der verwelkten Blätter wird Er neue, viel bessere als die vorigen wachsen lassen, die der Heilung all deiner Schmerzen dienen werden. 

Und jetzt, wenn selbst Dieser, der die Zügel von allem in dieser Welt hält, dir Seine Unterstützung und Seinen Segen verspricht, frage ich: Gibt es Raum für Zweifel? Nein, das ist genauso wahr, wie die Erde auf ihren Fundamenten steht und der Himmel sie wie ein Gewand mit seinem Segen bedeckt. Woran hast du oder jemand anderer im Hinblick auf den Segen Gottes zu zweifeln? An nichts. Ist nicht dein Leben selbst ein beständiges Zeugnis dafür, dass Gott immer barmherzig und gut ist? Wenn Er nicht allgütig gewesen wäre, hätte Er es dir dann gewährt, dass du denken, sehen, hören, handeln und das tun kannst, was du willst? Würde Er es dir gestatten, deinen Blick frei emporzurichten und Ihn wie einen Freund um Hilfe anzurufen? Wer von den irdischen Herrschern hat einmal seinen Untertanen solch eine Freiheit gegeben und hat sich so herabgelassen, dass sie immer Zutritt zu Ihm haben? Er selbst ist der Herr, der das getan hat und der überall ist. Sein Auge ist allsehend, und Er wägt alle Taten ab und schätzt sie ein. Aber ihr irrt euch, wenn ihr euch bemüht, die Werke Gottes zu beschleunigen. Kann eine Frau ihr Kind vorzeitig oder aber mehrere Kinder auf einmal gebären? Nein, das ist unmöglich. Wenn für die irdischen Angelegenheiten, wie die Geburt, eine bestimmte Anzahl von Tagen und Monaten erforderlich sind, damit die Anzahl der Zeit, die die Empfängnis erfordert, erfüllt wird, wie viel mehr erfordern dann die Werke Gottes völlige Beachtung? Aber ich sage: Eine Frau kann vorzeitig und verspätet gebären, beide Fälle sind jedoch tödlich für das Kind -- im ersten Fall ereilt der Tod das Kind, im zweiten schluckt der Tod das Kind noch im Inneren. Siehe, warum man das beachten soll. "`Wenn du jetzt den Inhalt Meiner Worte und die Gedanken Meines Geistes verstehst, wirst du in der Lage sein, Meinen Willen zu berücksichtigen"' -- so spricht Gott. 

Wenn du alle Zeichen dieses Buches kennen würdest, das geschrieben und vor dir aufgeschlagen ist und den ganzen Himmel und die ganze Erde beinhaltet, würdest du sehr deutlich die ganze Vergangenheit und die ganze Zukunft lesen und voraussehen können, die in ihm über diese und jene Welt gedruckt und geschrieben worden sind. Du würdest die Sprache der Dinge verstehen, die von sich selbst und von der Wahrheit zeugen.

Frage jenen Wurm, warum er kriecht, und er wird dir antworten, warum. Aber wirst du dieses Rätsel verstehen -- warum er ununterbrochen kriecht und was für einen Nutzen er davon hat? Nein, dir erscheint sein Kreuchen und Kriechen nutz- und sinnlos zu sein. Nein, Ich sage dir, dass es in sich so viel Inhalt und Wichtigkeit hat wie die Drehung der Erde um die Sonne. Ja, er verrichtet eine ehrbare Arbeit, obwohl er auf dem Boden erniedrigt ist. Frage ihn, warum er sie verrichtet, und er wird dir antworten. Aber wirst du, der du an den Worten Gottes zweifelst, an seine Worte glauben? Er kann manchmal das eine oder andere Blatt eines Baumes durchbeißen oder von der einen oder anderen Wurzel irgendeines Baumes von dir fressen; du wirst ihn jedoch dafür entschuldigen, denn das ist seine Arbeit. Und danke ihm dafür, dass er dir mit seiner offensichtlichen Unbekümmertheit viele Lektionen erteilt hat. Und siehe, er murrt nie, er ist immer dankbar. Wenn du ihn zertrittst und zerdrückst, nimmt er sein Schicksal mit Fassung an und nährt keinen Hass wegen des ihm zuteil gewordene Übels. Und wenn du ihn von seiner Stelle wegfegst, geht er mit Dankbarkeit zu einer anderen Stelle, indem er dir sagt: "`Mensch, ich habe dir nichts Böses getan; die Erde gehört Gott und ich erfülle meine wohl unangenehme Pflicht. Dir könnte ich als ein Übeltäter erscheinen, nicht jedoch meinem Schöpfer. Ich möchte dir sagen, meine Nahrung ist der Boden, und wie du siehst, ist er nicht allzu reich. Dennoch bin ich dankbar. Aber ich will dich ermahnen, dass ich dich auffressen werde, wenn du der Welt dienst, und merke dir -- sei mir nicht böse, wenn ich eines Tages mich einfinde, um in deinem Fett zu stochern. Wisse von jetzt an, dass es meine von Gott auferlegte Pflicht ist, mit allen so umzugehen, die sich in dieser Erde verstecken. Und wenn dir die Welt, so wie sie ist, lieb ist, und du dich vor meiner furchtbaren Gegenwart im Dunkel der Nacht trotzdem ekelst, wenn du dich in die Erde schlafen legst und dich ausruhst und nachdenkst, dann sage ich dir von jetzt an: Göttlicher Mensch, wenn du mich für einen Feind hältst, dann nimm die Flügel deines Heiligen Geistes, den Gott dir gegeben hat, und fliege nach Hause in den Himmel, denn da ist der beste und meist gesegnete Ort, dem sich weder ein Wurm noch eine Motte noch ein Übeltäter nähern können.


8. Juli 1900






\section{Siebtes Gespräch -- Beschluss}

"`Rufe aus voller Kehle, halte dich nicht zurück"'\footnote{Jes 58,1}, sagt der Herr. Wie lange noch werdet ihr zwei Sinnen dienen und zwischen zwei Vernünfteleien stehen? Wenn der Herr spricht, dann hört auf Seine Worte und seid nicht untreu, sondern treu. Lässt es sich noch deutlicher als auf diese Weise aussprechen? Versteht ihr noch nicht die Bedeutung eures Lebens? Wovor müsst ihr euch fürchten und vor wem erschrecken? Sind denn alle diese Menschen, die durch die Nase atmen, nicht sterblich? Sind sie nicht Spreu, die vom Wind verweht wird? Wenn sie heute sind, sind sie morgen nicht und sie verschwinden spurlos. 

Begreife also die Wahrheit, die ich dir in diesem Gespräch darzulegen habe. Ich bin Aphail, einer der dir dienenden Geister. Und der Herr hat mich angerufen und mich zu dir gesandt, um dir das mitzuteilen, was vollbracht werden soll. Ich komme vom Himmel, von der Wohnung Alphiola\footnote{Diesem Namen gibt Petar Danov dem Zentrum unserer Galaxie, in welchen nach der modernen Astronomie sich ein supermassives, schwarzes Loch befindet.}, von der Hauptstätte des Himmelreiches, wo alle Bitten und Gebete dieser Welt vor das Angesicht Gottes kommen. Weil du seit langer Zeit im Gebet verweilst, und dein Leben mit innerer Schwere und inneren Störungen belastest ist, will Gott dieses Geschwür von deiner Seele entfernen. 

Dieses Volk, um das es geht, hat eine innere Verwandlung zu durchlaufen. Es werden in der Ordnung Veränderungen geschehen, die Gott bald vollbringen wird. Eine gewisse Kraft wird durch dieses Land gehen. Ein Mensch wird von Gott kommen und die Wahrheit verkünden. In seinen Worten wird Kraft und Macht sein. Er wird ein Mensch sein, dessen Antlitz wie dieses eines Engels leuchten wird, in seinen Augen wird göttliches Feuer entzündet sein. Die heutigen Zeiten kündigen die künftigen Tage an. Deine Freunde, die Gott mit dir berufen hat, werden bis zum Ende treu standhalten, denn das ist der Wille des Herrn, der ihnen Gnade erweisen hat. Und für ihre Vervollkommnung und Annäherung an Gott arbeitet der Geist des Herrn in ihren Herzen. Die Morgenröte des Heiligen Geistes hat sie schon ergriffen, sie sind auf dem Weg der Erlösung. Der Herr hat auch in ihren Herzen jenes große Werk der Wiedergeburt zu vollbringen, über das zu dir selbst der Herr gesprochen hat. Wenn der Heilige Geist sie ergriffen und ganz erfüllt hat, wird in ihnen jene innere, große Verwandlung der göttlichen Geburt vorgehen -- eine Geburt aus dem Geist. In ihren Seelen gibt es noch viel Arbeit zu verrichten. Vor allem sollen ihnen wie beim Blinden, der zum Herrn ging, die Augen geöffnet werden --, auf dass sie innerlich den Ruhm Gottes, Seinen Ruhm, Seine Güte und Herrlichkeit ganz schauen. Sie sind immer noch Kinder im beginnenden göttlichen Leben, sie fürchten und ängstigen sich. Sie zweifeln immer noch. Die Schurkereien und Übeltaten des Teufels stören sie ununterbrochen und verfinstern ihren Verstand, damit sie die Worte des Herrn nicht völlig verstehen können. Der Teufel ist bemüht, Misstrauen und Zwietracht unter ihnen und innere Belastungen hervorzurufen sowie sie mit den Gütern dieser Welt zu blenden. Weil Gott jedoch der Mächtigste unter dem Himmel ist, wird Er sie von der Hand dieses Betrügers und Vaters jeder Lüge erlösen. Dabei gibt es auch andere Schwierigkeiten -- die Kirche selbst, die sich vom Geist Gottes entfernt hat und mehr dem Geist dieser Welt dient, wird ihnen auch zum Hindernis. Und wie Gott gesagt hat: "`Wehe euch Gesetzesgelehrten! Denn ihr habt den Schlüssel der Erkenntnis des Himmels weggenommen; ihr selbst seid nicht hineingegangen, und die hineingehen wollten, habt ihr gehindert."'\footnote{Mt 23,13-14.}

Du aber sollst dich in nichts von mir stören lassen, denn Ich spreche zu dir: Treu ist der Herr. Er selbst wird euch alle bald besuchen und euch zusammen segnen, damit ihr in jeder Tugend und jeder Güte Erfolg habt. Euer Leben soll sich ändern. Ängstigt euch nicht vor den dunklen Gewittern auf dieser Welt, denn sie sind Segensprüche Gottes. 

Ich habe auch den Auftrag vom Herrn, deinen Freunden, den Freunden des Herrn Jesus, mitzuteilen, sich selbst keine Hindernisse ins Leben zu legen. Ihre Wünsche nach all den guten und edelmütigen Taten werden erfüllt werden. Die großen Werke des Reiches Gottes lassen sich nicht so messen wie die Dinge in dieser Welt. Wenn ihre Herzen völlig befreit und dem Herrn ergeben wären, und wenn ihr Glaube so unerschütterlich und stark wäre, so würden sie Wunder wirken. Aber sie stolpern hier wie viele andere. Ja, Glaube -- er ist erforderlich. Ohne Glauben kann man Gott nicht wohlgefallen. Erwartet jede Kraft von Gott und fleht eifrig, und sie wird euch gegeben. Gott wird euch nichts Gutes vorenthalten, barmherzig ist Er und gütig.

Diese Dinge sind wichtig. Im Himmel wird bald ein feierlicher Tag kommen, an dem wir alle teilnehmen werden, und deshalb werde ich nicht zögern, an meine Stelle zurückzukehren und meine Verehrung und Hochachtung meinem Herrgott gegenüber zu äußern und Ihm mitzuteilen, dass ich meine Pflicht hier unten, wohin ich geschickt worden bin, treu erfüllt habe. Bis zu meiner Rückkehr werden große Ereignisse in dieser Welt geschehen, aber Ich werde zur vereinbarten Zeit wieder hier unten sein. 

Einer deiner Freunde, der dich am meisten liebt, hat mir aufgetragen, dich zu grüßen. Er erwartet bis jetzt die Erfüllung der Worte Gottes. Ich werde mich oben mit ihm wiedertreffen. Es ist Michael, einer der obersten Diener Gottes. Ich werde ihm vom Erfolg und Gang deines Lebens und des Lebens aller anderen Auserwählten Gottes erzählen. Michael erfreut sich sehr am Werk Gottes. Außerdem werde ich mich auch mit dem Diener Gottes Daniel, der zu deinen Freunden geschickt worden war, treffen und ich werde durch ihn von der Arbeit erfahren, die er verrichten sollte. Ananail\footnote{Es geht vermutlich um Anail oder Aniel -- einer der sieben Erzengel, welche über der Sphäre von Venus regieren. Seinen Einfluss bringt in den Kulturen die großen Handlungen der Liebe, welche für die Verbreitung des Christentums beigetragen haben. } freut sich über seinen bekannten Freund, den er liebt. "`Er ist so sehr in seine materiellen Angelegenheiten verstrickt, sagt er, "`dass er gezwungen ist, sein Leben vor großen Schwierigkeiten zu retten."' Und obwohl ich ihm ein paar Mal mitgeteilt habe, sich vor Unternehmungen, die Gott nicht wohlgefallen, zu hüten, hat er die gleichen Fehler wie ein Kind begangen. Aber ich hoffe und weiß gut, dass ihm das zur Lehre sein wird. Doch seine Entscheidung, von nun an Gott zu dienen, erfreut mich sehr.

Jetzt bin Ich überzeugt -- wenn sie alles lesen, was du ihnen vom Herrn schreibst, werden sie sich gemeinsam daran erfreuen. Gott wird euch mit den materiellen Mittel, die ihr benötigt, um Sein Werk zu vollbringen, versorgen. Kümmert euch um nichts, treu ist Gott, der eure Bedürfnisse kennt. Er wird euch alles zu seiner Zeit bereitstellen. Wenn ihr Seinen Segen habt, werden die Angelegenheiten günstig verlaufen.

Und jetzt, bevor ich wegziehe und mich in die himmlische Wohnstätte begebe, will ich dir mitteilen, dass die Fahnen Gottes weit und breit auf der Erde flattern werden. Es werden ständige Verhandlungen zwischen den irdischen Herrschaften geführt werden, aber letztendlich wird Gott sie heimsuchen. All diesen Raubtieren und Bestien, die heute herrschen, wird die Macht genommen werden. Das ist der Geist der Prophezeiung. Verlangt werden Elan und Kraft, damit ihre Herzen vom Essen und Trinken anderer irdischen Sorgen nicht ermüdet werden. Und vor allem, dient dem Herrn unermüdlich, habt Umgang mit Seinem Geist, und der Himmel wird an eurer Seite stehen. Für eure Erlösung, für einen, der für seine Sünden Reue zeigt, entsteht im Himmel eine große Freude unter den Engeln Gottes. Die Werke Gottes sind wunderbar -- unter ihnen gibt es eine solch enge Verbindung, wie zwischen zwei guten Geistern.

Ist für diejenigen, die nicht vom Geist erleuchtet worden sind, unser Kommen aus solch einem fernen Raum, um euch zu helfen, um eure Freuden, eure Mühsale manchmal zu teilen und euch zu trösten, euch ständig zu ermutigen, indem wir euch eine innere Überzeugung des Herzens bringen, dass der Herr all dies um eures Guten willen durch das Wirken Seines Großen und Heiligen Geistes tut, nicht verwunderlich? Wie viele Male sind wir euch entgegengekommen, um euer Leben zu retten, um eure Verzweiflung zu tilgen, um euch von einem tödlichen Gedanken zu erlösen. Wüsstet ihr nur, wie viele Male wir in euer Leben um des Herrn willen gekommen sind, um euch einen guten Ratschlag zu geben, euch auf einen guten Weg zu schicken und zu zeigen, dass dies der Weg des Herrn ist, dass dies Sein Wille und Sein Wunsch für euch ist! Und unser Lohn ist euer gutes und frommes Leben gewesen. Seht, das freut uns und wird uns immer freuen. Gibt es etwas Besseres als das? Nein, das ist die vollkommene Fülle. Wir sind immer nah, die Welt ist nicht wüst. Wir bewegen uns und leben in Ihm, denn das ist der oberste Wille Gottes. Wir dienen Ihm, wir lieben Ihn, wir lobpreisen Ihn. Wir überliefern Sein Wort von Geschlecht zu Geschlecht. Wir bewahren Sein Gesetz, führen Seine Gebote aus und sind immer bereit, alles zu tun, was Ihm wohlgefällt. Und kann es etwas Besseres zu tun geben als das? Nein. Den Ruhm Gottes zu schauen, über Seine Werke nachzudenken, Sein Gesicht zu sehen, das ist mehr, als ein unsterblicher Geist will. Und Gott spricht: "`Vater, ich will, dass alle, die du mir gegeben hast, dort bei mir sind, wo ich bin. Sie sollen meine Herrlichkeit schauen, die du mir gegeben hast, weil du mich schon geliebt hast vor der Erschaffung der Welt."'



9. Juli 1900, Novi Pazar





