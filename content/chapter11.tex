
	\chapter{Der Eigenwille}



Der Beginn der Weisheit ist das Verstehen des Weges des Herrn. Darin besteht der Anfang jedes Wissens, das für das menschliche Herz nützlich ist. Folglich, um weise zu werden, sollst du dich vom Eigenwillen lossagen, der die Haupthürde in jedem Leben ist. Der eigenwillige Mensch kann dasjenige, was gut ist, nicht zur Kenntnis nehmen, weil der Eigenwille der Beginn der Unordnung ist, die die Tugend ausschließt. Jeder, der im Leben fortschreiten oder Wissen und Weisheit von den Dingen in der Welt gewinnen möchte, soll sich vom Eigenwillen lossagen, der, wenn er nicht rechtzeitig aufgehoben wird, Unordnung bringt und das Leben selbst verstimmt. Denn der Eigenwille will auf seinen eigenen Wegen gehen, ohne über die schlechten Folgen nachzudenken. Der Eigenwillige möchte nach den Capricen seiner Wünsche die allgemeine Ordnung verändern und die Dinge dazu zwingen, sich gemäß seinem Willen zu bewegen. Für ihn gelten die Wünsche der anderen nicht. Er ist der Herr seiner Selbst, dessen Eigenwillen alle erfüllen sollen. Der eigenwillige Mensch ist hartnäckig auf all seinen Wegen. Und die Hartnäckigkeit seiner Seele ist der erste Beginn aller Unglücke, weil der eigenwillige, hartnäckige Mensch dem Guten der anderen gegenüber nachlässig ist, folglich auch seinem Guten, das auch das Gute seiner Nächsten einschließt. Aus Erfahrung wissen wir, dass die Nachlässigkeit die Mutter aller Übel ist. Und du, der du dieses Böse meiden möchtest, wisse, dass der Eigenwille deren Vater, und die Nachlässigkeit deren Mutter ist; die Hartnäckigkeit ist deren Bruder, und die Unvernunft deren Schwester. 

Und nun -- ist der Eigenwille aller Tugenden beraubt. Wer deshalb auf seinen Wegen geht und seine Ratschläge hört, wird nichts Gutes in seinem ganzen Leben erleben. Hüte dich vor solch einer schlechten Sitte, die das Böse in sich gebiert. Und wenn du weise genug bist, all diese Dinge zu verstehen, die dir gesagt werden, dann sei du selig, weil vom Begreifen der Dinge dein Glück abhängt; nur sie werden unmittelbar vom Geist Gottes beherrscht. Und nur dort, wo Gott selbst herrscht und die Dinge führt, kann es kein Böses geben, denn Er ist das vollkommene Licht, das alles durchdringt. Nimm deshalb an, dass Wissen und Weisheit dich leiten, und dein Leben wird immer mit Frieden und Freude erfüllt sein. Die Weisheit wird dich lehren, wie du leben sollst. Und das wirst du durch Beständigkeit und unermüdlichen Fleiß erreichen, die allmählich alle guten, von Gott gegebenen Eigenschaften entwickeln und sie deiner Seele geben werden. Mit ihrer Entwicklung und mit ihrem Blühen wird deine Seele einem bestellten und gut geordneten Garten gleichen, in dem alle Früchte deines Lebens rechtzeitig reifen und eine üppige Ernte aller Tugenden hervorbringen werden.

Und nun schreite mit der Fülle deines Herzens vor Gott und höre immer Seine Lehren, und sie werden dir ein ewiger Rat sein. Bewahre deine Seele von dieser schlechten Sitte und gib ihr keinen Platz, dass sie sich in dir einnistet, denn durch ihren Eintritt in deine Seele wird auch der Teufel eintreten, der Tausende von schlechten Samen säen wird. Und du wirst zum Pflanzgarten des Bösen werden und einen Gewinn für die Hölle hervorrufen, die über dich verfügen wird, wie sie möchte. Und weh dir, wenn sich der Teufel einschleicht und in dir einnistet, und die Hölle sich einen Weg zu deiner Seele bahnt. Ich sage dir die Wahrheit, du wirst nicht gesund herauskommen, solange du nicht mit deinem Leben bezahlst. Übrigens, hüte dich vor diesem unauslöschlichem Bösen, das dich aller Güter beraubt und dich für immer bedauernswert macht. Und nun, mach weder dem Teufel noch seinem Bruder Platz, halte seinen Vater und seine Mutter fern, damit sie nicht deine Nachbarn werden. Glaube mir, du wirst für ihre Speisen immer teuer bezahlen. Stärke dich deshalb mit der Tugend, kleide dich in die Wahrheit und bewaffne dich mit der Gerechtigkeit und nimm die Waffen der Liebe, und du wirst immer frei von ihrer Macht sein. Du wirst Freiheit haben, die dir niemand wegnehmen kann; du wirst Frieden besitzen, den niemand stören wird, und Glück, das dir niemand rauben wird.


\textit{	(Gedanken, die vom erhaben Geist diktiert wurden.) }

1903
