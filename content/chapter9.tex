	\chapter{Gedanken und Anleitungen}

Schlage niemals mit einer Arbeit über die Stränge. Habe es nie eilig, eine Arbeit bis zur Hälfte zu erledigen, denn dann wirst du sie wiederholen müssen, ohne das Ziel zu erreichen, das du verfolgst. Merke dir dabei, dass alles im Leben seinen Raum und seine Zeit haben soll, sonst wirst du es bereuen, dass du die Dinge nicht am richtigen Ort und zum rechten Zeitpunkt erledigt hast. Das wird dich viel kosten, es wird dir nämlich dasjenige entziehen, was dir zu empfangen zufällt, nämlich deine Zufriedenheit, dass du etwas Gutes getan hast. Denn die Schönheit eines Werkes hängt von der Richtigkeit und von der Ordnung ab, der gemäß es folgt, nämlich dass es zum richtigen Zeitpunkt getan wird -- und es ist dasjenige, was dem Werk seinen Wert verleiht. Biete deine Dienste nicht dort an, wo sie nicht benötigt werden, noch deine Freundschaft dort, wo sie keinen Platz hat. Bleibe an dem Ort, an den du verwiesen bist, gehe deinem Dienst nach, so wie du ihn verstehst und wie er dir von deinem Gewissen diktiert wird, denn darin besteht die Würde des Menschen. Jeder, der zu faul ist, um die Wahrheit zu begreifen, wird leiden, sein Leuchter wird erlöschen. Gib deshalb Acht, das zu vermeiden, nämlich in deinem Leben gegenüber den ewigen Prinzipien nachlässig zu werden und dein Glück auf Sand zu bauen. Bewahre dein Herz vor Eitelkeit, dann wirst du in deinem Leben glücklich sein. Begehre nicht dasjenige, was du nicht benötigst und verlange nicht nach dem Unmöglichen, denn damit begehrst du zwei unerreichbare Dinge. Rufe in jeder Angelegenheit Gott an, damit Er dich segnet, bevor du sie beginnst, dann wird dein Werk Erfolg haben.

Zweifle nicht am Weg der Wahrheit und frage nicht, welchen Nutzen du haben wirst, wenn du ihr folgst. Denn vernünftelst du, wirst du notwendigerweise in den Abgrund fallen und wegen deiner Tat, die Wahrheit missachtet zu haben, von allen vergessen werden.

Höre nicht auf, das Gute zu tun; denn hörst du auf, das Gute zu tun, dann sei gewiss, dass du in Kürze sterben und sofort spüren wirst, dass dein Leben kein Ziel hat. Wisse, dass das Ziel des Lebens die Tugend ist, und das Ziel der Tugend das Glück, und das Ziel des Glücks der Mensch, und das Ziel des Menschen die Wahrheit, und das Ziel der Wahrheit Gott, und das Ziel Gottes die Liebe, und das Ziel der Liebe das Ganze. Beherrschst du das Ganze, wirst du gewiss glücklich sein, denn du hast das Leben in seiner Fülle erhalten. Derjenige, der etwas erhält, erhält es von Gott, der der Schenker von allem ist. Sei nicht undankbar dem gegenüber, was auch immer es Ihm wohlgefällt, dir zu schenken, denn die Undankbarkeit ist die erste Sünde im Leben; sie verkürzt das Leben.

So, du hast die Weisheit jetzt erlangt, wisse, dass du nun auch die Tugend erlangt hast; die Tugend beschämt niemals, denn sie ist der gesunde Boden, auf dem du alles errichten kannst. Halte dich vom leeren Vernünfteln fern, denn das ist die Schlinge des Vaters der Lüge, mit der er dich berauben will. Erinnere dich daran, dass das leere Versprechen sind, die das Glück des Lebens vernichten. Und nun siehst du einen leeren Platz; setze nicht deinen Fuß dorthin, damit du nicht gefangen wirst, und es dir später schlecht wird. Wird es dir übel, dann wisse, dass du das Glück aus deinem Herzen erbrechen wirst. Verlierst du dasjenige, was du nicht rückgängig machen kannst, dann wisse, dass du verurteilt sein wirst zu leiden.

Und nun, wenn Gott spricht, dann höre mit deinem ganzen Herzen zu und halte sein Wort, damit du dich nicht der Weisheit beraubt findest und so hinausgeworfen wirst. Deshalb höre Demjenigen zu, der alles vollzieht. Er ist der ewige, unsichtbare Gott, dessen Wort sich in der ganzen Welt verbreitet. Hast du seine Worte des Lebens, bist du gerettet; die Rettung ist ein Werk der Liebe. Verachte nicht das, was die Liebe tut, sonst wirst du deiner Seele Schaden zufügen. Wisse: Mensch zu sein, bedeutet dasselbe wie Sohn Gottes zu sein, und um ein Sohn Gottes zu sein, ist es nötig, seine Wahrheit zu lieben. Denn das Ziel des Menschen ist die Wahrheit, das Ziel der Wahrheit das Wissen, das Ziel des Wissens die Aufklärung, das Ziel der Aufklärung die Veredelung und das Ziel der Veredelung die Erziehung, das Ziel der Erziehung die Wiedergeburt und das Ziel der Wiedergeburt ist die Tugend, und die Tugend ist der Grund von allem, denn sie ist ewig in der Ordnung der Welt. Wenn du sie besitzt, dann wirst du gegen alles geschützt sein. Wenn alles verschwindet und sich verändert, dann wird sie einzig in deiner Seele unveränderlich und unerschütterlich bleiben, um dich stark zu machen. Deshalb ziehe sie allen Reichtümern vor, unabhängig davon, wie und wo sie sind, denn ziehst du sie vor, wird sie dich mit der Fülle ihres Herzens lieben und ihre Liebe wird dich stark machen und dich während deines ganzen Lebens trösten.

Wisse, dass die zwei ewigen Ziele des Lebens Tugend und Liebe sind. Derjenige, der sie geboren hat, ist der Vater von allem; seinen Namen weiß niemand. Und nun, begreife die Wahrheit, dass das Ziel des Lebens die Geburt ist, das Ziel der Geburt die Liebe, das Ziel der Liebe das Leiden, das Ziel des Leidens der Charakter, das Ziel des Charakters die Tugend, das Ziel der Tugend Gott, das Ziel Gottes die Liebe, und die Liebe ist das Endziel der grenzenlosen Ewigkeit, in der alles in Frieden und in Freude ist; in Frieden, denn der ewige Wille des unerreichbaren Emanuel, der stärkende Gott, herrscht; in Freude, denn Tugend und Liebe herrschen. Folglich, um leben zu können, wie es nötig ist, bedarf es dieser zwei Ursprünge des Lebens, die es erfrischen und erheben, denn ohne ihren Beistand kann es kein Leben geben. 

Bewahre deine Seele vor leerem Geschwätz, denn darin liegt unvermeidlich die Sünde; und dort, wo Sünde ist, kann es keinen Frieden geben. Folglich -- jeder, der den Frieden und die Freude wünscht, soll die Sünde meiden. Denn sie zerstört all das Gute; und dort, wo das Gute zerstört wird, entsteht Unglück. Denn das Böse und das Gute sind miteinander unvereinbar -- ziehst du die Sünde vor, entsteht Unglück, denn das sind zwei Dinge, die miteinander in einem verbunden sind als Ursache und Wirkung; mit dem Empfangen des einen wirst du auch das andere empfangen. Nimmst du das Gute wahr, wirst du auch das Glück wahrnehmen, denn das Ziel des Guten ist das Glück. Und nun -- wisse deshalb, dass in der ewigen Ordnung jede Zerstörung ihre Folgen hat. Wird dein Frieden zerstört, dann hast du ein Gebot des Guten verletzt. Deshalb sieh zu, dass du schnell deine Fehler berichtigst, denn sonst wird deine Wunde größer werden.

Hör auf die Stimme deines Schutzengels, deines Gewissens, um immer ruhig und selig zu sein. Achte deshalb darauf, wie du auf dem Weg des Lebens gehst. Schwanke nicht, wenn du zwischen zwei gegensätzlichen Wegen wählen sollst, denn in der Schwankung wird sich deine Schwäche zeigen; sie kann dir zu einer Falle werden, durch die du zu Schaden kommen kannst. Zwischen zwei Wegen wähle immer den geraden Weg, auch wenn er schmal ist und es wenige sind, die ihn eingeschlagen haben. Sei überzeugt, dass du Zeit sparen wirst, weil du auf keine großen Hürden treffen wirst. Vernünftle nicht mehr, als du weißt, denn sonst wirst du einem Menschen gleichen, der mehr Geld ausgibt, als er hat. Das ist dasselbe, als ob du nach Armut oder Wahnsinn suchen würdest. Und nun -- achte darauf, was du sprichst, damit deine Zunge nicht ausrutscht und das sagt, was du nicht weißt -- viel verlierst du durch ein solches Ausrutschen, das du nachher schwer wieder gutmachen kannst.

Suche die Ruhe in der Zurückgezogenheit, wenn du dich geistig ermattet fühlst, wo deine Seele Nahrung zu sich nehmen kann. Sei nicht geistig faul zu beten -- das Gebet ist die Kraft des Lebens. Schiebe dasjenige, was Gott dir zu tun befiehlt, nicht auf, und weise dasjenige, was Gott will, nicht zurück, wenn du das Gute deiner Seele wünschst. Welchen Nutzen wirst du vom Aufschieben und welchen vom Zurückweisen haben? Weise das Schwanken und die Hartnäckigkeit auf dem Weg deines Lebens zurück, damit die Wahrheit in dir lebe, denn Gott lebt nur in demjenigen, der sich vor ihm in Wahrheit und Reinheit bewegt. Lerne, dich vor Gott mit demütigem Herz zu bewegen, und die Liebe des Lebens wird in dir leben.

Enthalte dich allem, was schädlich ist -- versuche nicht, Dinge zu erwerben, die dir zu nichts Gutem dienen. Sei immer wachsam und hüte dich davor, die günstigen Momente im Leben nicht zu versäumen, weil du sie später nicht wieder erlangen kannst. Das, was versäumt wurde und nicht erreicht werden kann, ist ein wahrer Verlust auf dem Lebensweg. Und nun, höre immer jene innere Stimme, die zu dir spricht und dich leitet, denn aus ihrem Sprechen wirst du erfahren, was du tun sollst, um die wahre Seligkeit zu erlangen. Sei in diesen drei Dingen nicht faul, wenn du dasjenige, was du hast, nicht verlieren willst: Gib den Glauben nicht auf, vergiss die Hoffnung nicht, lösche die Liebe nicht aus, denn in ihnen liegt das Leben der Zukunft verborgen. Zerstöre deshalb nicht das Fundament deiner Seele, ohne das du nicht als Mensch existieren kannst. Wisse, dass der Leuchter deiner Seele die Liebe ist, die dich mit ihren beiden Händen stützt, damit du nicht in den Abgrund der Verzweiflung fällst, aus dem es kein Zurück gibt, wenn du dorthin allein gelangst. Deshalb denke richtig, bevor du den Weg des Diesseits einschlägst, damit du nicht dorthin gelangst, wo du nichts anderes finden wirst als Leere, Gram und Verzweiflung.

Von zwei Gedanken empfange immer den ersten, erlaube dem zweiten nicht, sich durchzusetzen, denn dann wirst du notwendigerweise einen Fehler begehen. Zweifle nie, wenn du eine Arbeit beginnst, denn dort, wo es Zweifel gibt, gibt es auch Entzweiung. Die Entzweiung wird in deiner Seele notwendigerweise Unstimmigkeiten in den Bestrebungen hervorrufen, die das Erreichen deines edlen Zieles, das du dir gesetzt hast, verhindern werden. Diene nicht zwei Verstandesformen, denn das ist unmöglich. Schütze deine Gedanken, wenn du eine Arbeit ausführst. Sieh zu, dass der Teufel nichts davon erfährt. Gib nicht an, sprich nicht über dich selbst, denn das wäre verlorene Zeit. Mache keine Pläne für dein Leben, denn das ist nicht deine Angelegenheit -- darum kümmert sich ein anderer. Tue nur das, was dir gegeben wird, und das Ende wird gut sein. Mache einem schlechten Gedanken keinen Platz, damit er nicht in deine Seele eindringt. Verberge deine Fehler nicht vor dir, entschuldige dich nicht für das, was du getan hast, weil du damit zwei Fehler begehen würdest. Die Einheit eines jeden Gedankens und Gefühls sind die Kraft, die zum Ziel führt. Setze dir keine verführerischen Dinge zum Ziel deines Lebens, damit sie dich nicht vom Weg deiner Berufung abbringen. Lasse dich nicht vom Äußeren der Dinge verführen, weil du notwendigerweise Fehler in der Wahl begehen wirst. Denke nicht, dass alles, was dir geboten wird, zu deinem persönlichen Wohl ist. Die Dinge haben nichts mit diesem oder jenem Glauben gemeinsam, sie machen dein Leben weder besser noch schlechter -- der Gebrauch der Dinge ist dasjenige, was jedermanns Glück zerstört. Die erste Sache im Leben ist zu lernen, Gebrauch von den Dingen zu machen und jedes Ding, das gebraucht wird, soll einen gewissen Nutzen erbringen, ansonsten ist es ohne diese gute Frucht schädlich. Wünsche nicht, mehr zu tragen, als deine Kräfte erlauben, weil du notwendigerweise durch die Überanstrengung zu Schaden kommen würdest. Keine Übertreibung ist recht und keine Übertreibung kann einen Nutzen bringen. Tue alles, was du tust, so als ob es für dich wäre, denn sonst kannst du mit anderen zusammen zu Schaden kommen. Versprich nicht das, was du nicht halten kannst; rühre nicht die Dinge an, die du nicht verstehst. Wenn dich ein Unglück ereilt, kümmere dich nicht darum und murre nicht, denn begehst du diesen Fehler, wirst du mit zwei Bürden entlohnt werden. Freue dich nicht, wenn dich das Glück ereilt, denn sonst wirst du zwei Unglücke anziehen -- du kannst das Glück verlieren und das Glück kann dich täuschen. Das Glück gleicht einer Frau, die allen zulächelt, als ob sie sie lieben würde, aber wenn die Zeit zum Heiraten kommt, verschwindet sie. Wenn dir das Glück zulächelt, denke nicht, dass es dir Gutes wünscht -- nein, es täuscht dich. Wenn es zu dir über die Liebe spricht, denke daran, dass es zu dir über das Spiel Faulsein spricht. Wer an das Glück glaubt, vergeudet seine Jugend. Sei nicht dumm, sei nüchtern; das Leben ist weder Spiel noch ein täglicher Scherz -- es erfordert Arbeit und Beständigkeit in der Tugend.

1903
