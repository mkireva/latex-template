\chapter{Der gute Zeitpunkt}

Was willst du? Sage es deutlich, schwanke nicht. Ob du empfangen wirst oder nicht, suche nicht -- jedes Bedürfnis soll gestillt werden und gewiss rechtzeitig. Bittest du, dann bitte richtig, wenn du erhalten möchtest. Denke nicht, dass du ohne Bitten erhältst, was du möchtest. Der Wunsch soll das Bedürfnis offenbaren, und das Bedürfnis soll richtig gestillt werden, damit kein Bruch in der Ordnung der Seele geschieht. Wenn du bittest und nicht bekommst, so wisse, dass du die Art und Weise des Bittens begreifen und nicht verstummen sollst. Schweige nur dann, wenn du satt bist, aber nicht dann, wenn du hungrig bist -- das Ziel des Schweigens ist es, der Seele Zeit zu geben, damit sie ihre Pflicht erfüllen kann. Und wisse, zwei Dinge gleichzeitig zu tun, ist unzweckmäßig -- jeder Moment im Leben hat seine Arbeit, wo nichts anderes getan werden darf, als dasjenige, was gerade nötig ist. Und verstehe, dass Gott der Gott der Ordnung und nicht der Unordnung ist; in ihm existieren keine Unzweckmäßigkeiten. Er tritt ursprünglich aus, Er tritt ursprünglich ein und wirkt ursprünglich aus der Ordnung in der Ordnung. In Ihm gibt es weder Sprünge noch Leeren -- in Ihm ist alles voll und für jeden Moment der Ewigkeit bereit. Er kann von nichts überrascht werden, weil Er in allen Momenten auf ewig ist.

Und nun folgt jedes Ding von Anbeginn seinem gegebenen Moment, und dieser Moment, der nicht an seinem richtigen zeitlichen Beginn empfangen wurde, wird einen Abdruck der Unordnung in dieser Seele hinterlassen, welche nicht treu an ihrem Posten stand, um alle gegebenen Momente des Lebens zu empfangen. Und jeder vorzeitig verlorene Moment ist für immer für die Ewigkeit verloren. Nun erinnere dich, dass es keine Ausrede dafür gibt, dass du nichts verstanden hast. Deshalb sollst du fragen, damit du verstehst, bereit zu sein. Deshalb sollst du ständig beten, damit du mit den Momenten, die du erwartest, in Verbindung bist, bereit wie der Bräutigam für die Braut, um dich in die bestimmten Zeit loszumachen. Stehe deshalb bis zum Ende treu an deiner Stelle. Nichts darf dich verführen, früher oder später aufzugeben, denn in solch einem Fall würdest du außerhalb der göttlichen Ordnung sein. Alles, was außerhalb der göttlichen Ordnung ist, kann nicht existieren noch im Einklang mit ihm wirken. Jedes außerhalb von seiner Ordnung stehende Ding hört auf, ihm zu gehören. Denke nicht, dass du in den Himmel eintreten kannst, wenn du willst und wenn du es wünschst. Nein, dort kannst du nur dann eintreten, wenn du gerufen wirst und für den gegeben Moment bereit bist. 

Und nun, hüte dich davor, die Gelegenheit deiner Berufung zu verpassen, denn sie ist nur eine einzige in der ganzen Ewigkeit. Und wisse, dass die Ewigkeit zu ihrer Grenzenlosigkeit eilt, in der das Leben entsteht. Und alles, was einmal geboren wurde, soll demjenigen folgen, der ihn geboren hat. Denn wie das Kind nicht ohne Mutter und Vater sein kann, so ist deine Seele auch ohne Gott undenkbar -- Er ist der Anfang und das Ende von allem, und ohne Seine Gegenwart gibt es kein Leben. Und nun wenn du das Leben verstanden hast, dann hast du die gegebene Ordnung begriffen und die gegebenen Momente ergriffen. Folglich schreitest du vorwärts zum Ziel hin, die die Einheit von allem in der göttlichen Ordnung ist. Dort treten alle bewusst ein in die allgemeine Verkündigung der Liebe, die alle Wesen geboren hat, vom grenzenlos Kleinen bis zum grenzenlos Großen, vom Menschen bis zum Erzengel und bis zum grenzenlos Großen. 

Und nun, erinnere dich -- es gibt nur einen gegebenen Moment, in dem du geboren wirst und in die Fülle des Lebens eintreten kannst, denn in der ganzen Ewigkeit gibt es keine zwei gleichen Momente noch zwei gleichförmige Seelen. Denn der Moment unterscheidet sich vom anderen Moment kraft seines Gewichtes, und die Seele unterscheidet sich kraft ihres Bewusstseins. Und nun, denke nicht leichtsinnig über die Dinge nach, dass Gott so handeln kann, wie Er möchte -- bei Gott gibt es keine Wünsche wie beim Menschen. In Ihm gibt es Liebe, die jedem dasjenige geben will, was ihm zusteht. Und dasjenige, was Gott gibt, soll er in seiner bestimmten Zeit geben oder zu dem bestimmt gegebenen Moment.

Verstehst du, wo du jetzt bist? Wenn du es verstehst, dann denke nach! Denn die Momente kommen, und du bist selig, dass Gott mit ihnen ist. 

1903
